\section{Feststoff-Gasphasen Reaktionen}
(Nicht katalytische) Reaktionen zwischen Feststoff und Gasphase kommen in der Praxis oftmals vor:
\bi
\item R�sten von sulfidischen Erzen (mit Sauerstoff)
\item Reduktion von Eisenerz (mit Kohlenmonoxid)
\item Vergasung von Kohle zu Syngas
\item Absorption von SO$_2$ durch Kalk
\item Brennen von Kalkstein zu gebranntem Kalk
\ei
\subsection{Por�ser Feststoff (CVI)}
Durch gr��ere spezifische Oberfl�che wird die Reaktionszone gr��er und es kommt zu Massentransportph�nomenen wie bereits bekannt. Die Konzentrationsgradienten h�ngen hierbei direkt mit Reaktionsrate und Diffusion zusammen.\\
Bei Reaktionen ist zwischen 2 F�llen zu unterscheiden:
\be
\item Kleine Poren, Gro�es Korn: Gradient des Edukts zur Kornmitte hin, diffusionsgehemmt.
\item Gro�e Poren, Kleines Korn: Quasihomogene Eduktkonzentration im Korn, Reaktion ist limitierend.
\ee
In diesem Fall spricht man von Chemical Vapor Infiltration ({\bf CVI}).\\
Die Reaktion l�uft in folgenden Schritten ab:
\be
\item Reaktion in der Gasphase von Precursor zum Edukt
\item Filmdiffusion
\item Porendiffusion
\item Oberfl�chenreaktion mit Filmbildung
\ee
R�ckweg analog.

\subsection{Nicht-Por�ser Feststoff}
Durch definierte Reaktionsoberfl�che wandert die Reaktionszone von au�en nach innen durch den Feststoff.\\
Bei der Reaktion ist zwischen 2 F�llen zu unterscheiden:
\be
\item Reaktion bildet ausschliesslich gasf�rmige Produkte: Radius nimmt mit der Zeit ab
\item Reaktion bildet ebenfalls festes Produkt: Das entstehende Produkt \emph{muss} por�s sein. Hier muss die Diffusion durch diesen Bereich betrachtet werden. Hierzu bietet sich das \emph{Shrinking Core Model} (SCM) an.
\ee
Im Fall 2 spricht man von Chemical Vapor Deposition ({\bf CVD}).\\
Die wichtigsten Schritte sind:
\be
\item Konvektiver Transport des Edukts
\item Massentransport zum Substrat (Grenzfilm)
\item Adsorption (aus dem Film)
\item Oberfl�chendiffusion, Reaktion und Bildung des por�sen Produktes
\ee
R�ckweg analog. Der geschwindigkeitsbestimmende Schritt ergibt sich hierbei durch:\\
\begin{minipage}[t]{.45\linewidth}
{\bf Kinetische Limitierung}
\[ \frac{1}{k_s} \gg \frac{\delta}{D} \]
Schnelle Diffusion
\bi
\item Hohe Gasgeschwindigkeit
\item Niedrige Temperatur
\item Geringer Druck
\ei
\end{minipage}
\hfill
\begin{minipage}[t]{.45\linewidth}
{\bf Massentransportlimitierung }
\[ \frac{1}{k_s} \ll \frac{\delta}{D} \]
Langsame Diffusion
\bi
\item Geringe Gasgeschwindigkeit
\item Hohe Temperatur
\item Hoher Druck
\ei
\end{minipage}

\subsection{Zusammenfassung}
Der CVD/CVI-Prozess muss durch die Reaktion gesteuert werden, um Kontrolle �ber die Vollst�ndigkeit der Reaktion zu bekommen. Hier kommt es auf das Verh�ltnis von Depositions- und Diffusionsrate an.

