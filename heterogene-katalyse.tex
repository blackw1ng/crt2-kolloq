\section{Heterogene Katalyse}
\subsection{Oberfl�che des Katalysatorkorns}
Wie in der homogenen Katalyse ist eine spezielle {\it geometrische} und {\it elektrochemische} Form der Oberfl�che notwendig, um gute Reaktivit�t und Selektivit�t zu erzeugen.

\subsubsection{Ensemble-Effekt}
Ein \emph{Ensemble} ist die Anzahl an Zentren, die f�r eine optimale Reaktion ben�tigt werden. Man unterscheidet zwischen:
\bi
\item {\bf On-Top} Ein Molek�l adsorbiert an einem aktiven Zentrum
\item {\bf Bridge-Bonded} Ein Molek�l ben�tigt zwei benachbarte Zentren
\item {\bf Hollow-Site} Ein Molek�l adsorbiert zwischen drei Zentren
\ei

\subsubsection{Liganden-Effekt}
Man spricht von \emph{Ligand-Effect}, wenn die Eigenschaften eines Ober\-fl�chen-Atoms durch Austauschen eines gleichartigen Nachbar-Atoms durch ein Andersartiges ver�ndert werden.

\subsubsection{Einflu� von Oberfl�chendefekten}
\bi
\item Stufen und Ecken sind aktiver
\item Adatoms (einzelnes Atom auf einer Fl�che) sind besonders aktiv
\item Zusammenfassend passiert an Unregelm��igkeiten am meisten.
\ei

\subsubsection{Zukunft}
Derzeit unterscheiden sich die Oberfl�chenforschung und die Katalysatorenforschung stark. 
W�hrend bei der Oberfl�chenforschung auf ideale Bedingungen, atomare Betrachtungsweise und Erforschung von Fundamentalmechanismen fokussiert wird, wird bei der heterogenen Katalyse stets mit irealen Bedingungen, System-Betrachtungsweise und dem Fokus auf Aktivit�t, Selektivit�t, Stabilit�t und Effektivkinetik gearbeitet. Das Ziel hierbei ist eindeutig die L�cke zu schlie�en.\\
In der (aufsteigenden) Komplexit�t ist zwischen chemischer
\bi
\item Ad-/Desorption
\item Simple Reaktion
\item Selektivit�t
\ei
und Strukturkomplexit�t
\bi
\item Monokristallin
\item Polykristallin
\item Modellkatalysatoren
\item Realit�t
\ei
zu unterscheiden.

\subsubsection{Methoden zur Strukturanalyse}
Hier kann zwischen Volumenmethoden
\bi
\item Na�chemische/UV: Anzahl / Art der Zentren, Oxidationszahl
\item Single Pellet Apparatus: $D_E$
\item Conductometric Detrimination: $\lambda_E$
\item NMR / ESR: Bindungart
\item R�ntgendiffraktometrie: Kristallstruktur
\ei
und Oberfl�chenmethoden
\bi
\item IR-Spektroskopie: Oberfl�chenbeschaffenheit
\item Temperatur Adsorption: Anzahl / Art aktiver Zentren
\item BET / Porosimetrie: Spezifische Oberfl�che, Porengr��enverteilung
\item Secondary Ion Mass Spectroscopy (SIMS): Elementverteilung
\item XPS: Bindungsart
\ei

\subsubsection{Trends}
Die Kombination von molekular definierten Zentren mit heterogenem Support stellt einen aktuellen Trend dar. Hier kann eine erh�hte Selektivit�t durch die spezifische Geometrie des Supports erreicht werden.

\subsection{Porendiffusion}
F�r die Betrachtung des Einflusses der Porendiffusion auf den Massentransport im Pellet wird, wie gewohnt zur Betrachtungsweise via Massenbilanz gegriffen. Hierbei wird vereinfachend angenommen:
\bi
\item Keine Filmdiffusion
\item Isothermie
\item Gleichartige Porenstruktur
\item Spherisches Katalysatorkorn
\item Pseudo-Homogenes System
\item Simple Reaktion
\ei
Ausgehend von der Massenbilanz:
\[ \eta = \frac{4 \pi R^2 D_e \frac{c_s}{R} \left( \frac{df}{dx} \right)_{x=1} }{\frac{4}{3} \pi R^3 k c_s }  \quad \mbox{mit} \quad \left( \frac{df}{dx} \right)_{x=1} = \frac{\phi}{\tanh \phi} - 1\]
ergibt sich
\[ \fbx{\eta = \frac{3}{\phi} \left( \frac{1}{\tanh \phi} - \frac{1}{\phi} \right) } \]
Wie gewohnt: F�r $\phi > 3$ ist $\tanh \approx 1$. Somit: $\eta = \frac{3}{\phi}$.
