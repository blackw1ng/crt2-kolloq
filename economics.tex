\section{Wirtschaftliche Betrachtung von Prozessen}
\subsection{Produktionskosten}
\bi
\item Rohstoffe, Additive, Katalysatoren, etc.
\item Energie (Elektrizit�t, Dampf, K�hlmittel, etc.)
\item Personalkosten (L�hne, Geh�lter, Pr�mien, etc.)
\item Kapitalkosten (Zinsen und Abschreibungen)
\item Erhaltungskosten
\item Umweltschutz (z.B. Abwasserbehandlung)
\item Analyse und Qualit�tsmanagment
\item Versand und Logistik
\ei

\subsection{Andere Kosten}
\bi
\item Generalia: Beteiligung an Gesamtkosten des Unternehmens
\item Marketing: technisches Marketing, Publikationen, Werbung
\item Forschung und Entwicklung: Investition in die Zukunft!
\ei

\subsection{Kalkulation}
Preise f�r Hauptpositionen (Reaktor, W�rme�bertrager, etc.) sind normalerweise bekannt.
Sind die Kosten $P$ f�r gew�nschte Gr��e oder Kapazit�t $C$ nicht bekannt kann die Korrelation
\[ \frac{P_1}{P_2} = \left( \frac{C_1}{C_2} \right)^m \]
mit einem Degressionskoeffizienten von z.B. $m=0,66$ zur Hochrechnung herangezogen werden.\\
Direkte Nebenpositionen (Montage, Verrohrung, MSR-Technik) und indirekte Nebenpositionen (Planungskosten, unerwartete Ereignisse) werden mit Faktoren abgesch�tzt und betragen in etwa das $1,9$ bis $3,5$-fache der Hauptpositionen.

\subsection{Kostendiagramm}
Linearer oder nichtlinearer Auftrag von Erl�sen bzw. Kosten [$EUR/a$] gegen�ber Auslastungsgrad $X$.
\bi
\item fixe Kosten: Kapital, Personal...
\item variable Kosten: Rohstoffe, Energie...
\item Nutzenschwelle: Erl�se > fixe + variable Kosten
\item Stillegungspunkt: variable Kosten > Erl�se
\ei

\subsection{Kapitalrendite und Amortisationszeit}
Der Parameter {\it return on investment} $r$ (ROI) basiert auf der Annahme, dass Profit und Abschreibungen verwendet werden, um die Investitionskosten zur�ckzuzahlen.
\[ r = \frac{ \mbox{Profit}/a+ \mbox{Abschreibungen}/a }{ \mbox{investiertes Kapital} } \cdot 100 = \frac{1}{t_r} \quad \left[ \%/a \right] \]
Wird f�r die Finanzierung Eigenkapital anstelle von Fremdkapital herangezogen, werden im Z�hler noch die j�hrlichen Zinsen hinzugerechnet.
In der chemischen Industrie sind gew�hnlich Amortisationszeiten $t_r$ von weniger als vier Jahren gefordert, um die wirtschaftlichen und technischen Risiken zu rechtfertigen. Nur f�r gut etablierte Massenprodukte mit einem sicheren, wachsenden Markt sind l�ngere Amortisationszeiten m�glich.

\section{Prozessentwicklung}
Unterschiede in der Prozessentwicklung f�r Massen- und Feinchemikalien sind durchg�ngig.\\
Scale-Up als empirischer Prozess ist zeitraubend und sicherheitstechnisch riskant, was zu einer Bevorzugung voraussagender Modelle f�hrt.
\subsection{Hauptschritte in der Prozessentwicklung}
\be
\item {\bf Entdeckungsphase} eines neuen Produktes, Katalysators, Syntheseweg
\item {\bf Verfahrenskonzept:} Machbarkeit, Wirtschaftlichkeit, Reaktorkonzept, Trennverfahren
\item {\bf Miniplant/Pilotanlage:} Entwicklung des Marktes, Best�tigung der Machbarkeit, Erforschung von Langzeiteffekten
\item {\bf Kommerzielle Anlage:} Detailplanung und Montage 
\ee
Zwischen den Hauptschritten steht eine Bewertung, die zum Verwerfen der Entwicklung, zur Wiederholung eines Schrittes oder zur Fortsetzung der Kette f�hrt.\\
Moderne Prozessentwicklung geschieht aus Zeitgr�nden oftmals parallel ({\it simultaneous engineering}).
Dies erfordert eine intensive Zusammenarbeit zwischen unterschiedlichen Disziplinen (Chemiker, Reaktionstechniker, Trenntechniker, etc.).
