\section{Homogene Katalyse}
\subsection{Allgemeines}
Als Basis dient der Vergleich zwischen homogener und heterogener Katalyse aus dem Kernfach:
\begin{table}[H]
\begin{tabular}{l|c|c}
                	& \bf Homogen   & \bf Heterogen \\ \hline
Aktivit�t       	& hoch          & variabel \\
Selektivit�t		& hoch          & variabel\\
Reaktionsbedingungen	& mild          & hart\\
Lebensdauer     	& variabel      & lang\\
Vergiftungsgefahr  	& niedrig       & hoch\\
Diffussionsprobleme     & niedrig       & hoch\\
Kosten f�r Regeneration	& hoch          & null\\
\end{tabular}
\end{table}
Spezielle Aspekte sind hierbei:
\bi
\item Meist �bergangsmetallkomplexe ($Pd, Rh, Ni, Co$) $\rightarrow$ Metallkomplexchemie
\item Viel Auswahl an Liganden, Oxidationszahlen und Koordinationen m�glich
\item Taylor-Made-Catalysators
\item Deswegen aber auch nur geringe Temperaturen m�glich
\ei
\subsection{Katalytischer Kreislauf}
\begin{center}
\bild[width=65mm]{images/catcycle.eps}
\end{center}
Hierbei haben die Teilschritte auch besondere Bezeichnungen
\bi
\item
\begin{minipage}{.5\linewidth}
\bild[scale=0.65]{images/assoziation.eps}
\end{minipage}
\begin{minipage}{.5\linewidth}
$\rightarrow$ Assoziation / Koordination\\
$\leftarrow$ Dissoziation
\end{minipage}

\item 
\begin{minipage}{.5\linewidth}
\bild[scale=0.65]{images/addition.eps}
\end{minipage}
\begin{minipage}{.5\linewidth}
$\rightarrow$ Oxidative Addition \\
$\leftarrow$ Reduktive Elimination
\end{minipage}

\item
\begin{minipage}{.5\linewidth}
\bild[scale=0.65]{images/insertion.eps}
\end{minipage}
\begin{minipage}{.5\linewidth}
$\rightarrow$ Insertion\\
$\leftarrow$ Extrusion
\end{minipage}

\item
\begin{minipage}{.5\linewidth}
\bild[scale=0.65]{images/coupling.eps}
\end{minipage}
\begin{minipage}{.5\linewidth}
$\rightarrow$ Oxidative Coupling\\
$\leftarrow$ Reductive Cleavage
\end{minipage}
\ei

\subsection{Charakterisierung von Katalysatoren}
\subsubsection{Turnover Frequency (TOF)}
\[ \fbx{ TOF = \frac{\mbox{Bildungsgeschwindigkeit}}{\mbox{Katalysatorkonzentration}} = \frac{R_j}{c_{Kat}} } \]
\subsubsection{Turnover Number}
\[ \fbx{ TON = \frac{\mbox{Max. Menge Produkt bis Kat. deaktiviert}}{\mbox{Menge Katalysator}} = \frac{n_{Prod,max}}{n_{Kat}} } \]
\subsubsection{Productivit�t}
�hnlich wie TON, pro Katalysatormenge in bestimmter Zeit umgesetzte Masse 
\[ \fbx{ P = \frac{m_{Prod}}{m_{Kat} \cdot t} } \]
\subsubsection{Selektivit�t}
\[ \fbx{  S_{ki} = \frac{\mbox{Menge an gebildetem Produkt k}}{\mbox{Menge an umgesetztem Reaktanden i}} = \frac{Y_{ki}}{X_i} = \frac{\dot{n}_k - \dot{n}_{k0}}{\dot{n}_{i0} - \dot{n}_i} \frac{\left|\nu_i\right|}{\left|\nu_k\right|} } \]

