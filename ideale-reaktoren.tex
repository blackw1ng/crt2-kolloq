\section{Ideale Reaktoren}
Allgemeine Massenbilanz:
\[ \fbx{ \underbrace{ \frac{\partial c_i}{\partial t} }_{ \mbox{Akk.} } = \underbrace{-\div \left( c_i \bar u \right)}_{ \mbox{Konvektion} }
+ \underbrace{\div \left( D_{e,i} \grad c_i \right)}_{ \mbox{Dispersion} } + \underbrace{\sum_j \nu_{ij} r_j}_{ \mbox{Reaktion} } } \qquad \left[ \frac{kmol}{m^3s} \right] \]

\subsection{Ideales Str�mungsrohr}
\[ \fbx{\frac{\partial c_i}{\partial t}=-u \frac{\partial c_i}{\partial z} + \Gamma_i} \qquad DaI=\frac{\tau r_0}{c_{10}}=\int^{X_i}_{0} \frac{r_0}{r} dX_i \]
mit $r=k c^n$ und volumenkonstanter Reaktion:
\[ \frac{r_0}{r}= \frac{k c^n_{10}}{c^n_1} = \frac{1}{\left(1 - X \right)^n} \]
f�r $n=1,2,3...\ \Rightarrow$ hyperbolische Funktion; f�r $X \rightarrow 1 \ \Rightarrow \frac{r_0}{r} \rightarrow \infty$ \\
Fl�che unterhalb Hyperbel entspricht $DaI$.
\bi
\item[$\Rightarrow$] Reaktionen h�herer Ordnung ben�tigen gr��ere $DaI$-Zahlen (l�ngere Verweilzeit oder gr��eres Reaktorvolumen, um den selben Umsatzgrad zu erreichen.
\ei
\paragraph{Spezialf�lle}
\bi
\item {\bf reversible Reaktion} Asymptote sinkt von $X=1$ auf $X=X_{eq}$
\item {\bf autokatalytische Reaktion} $r/r_0$-Kurve weist Minimum ($r=r_{max}$) auf
\ei

\subsection{Idealer kontinuierlicher R�hrkessel}
\[ \fbx{V \cdot \frac{dc_i}{dt}=\dot V \left( c_{i0} - c_i \right) + V \Gamma_i} \qquad  X_i = \frac{V r}{\dot V c_{i0}}\frac{r_0}{r_0} = DaI \frac{r}{r_0} \]
Rechteck zwischen $X$-$r/r_0$-Schnittpunkt und Achsen entspricht $DaI$.\\
$\Rightarrow$ Gew�hnlich ist $DaI_{CSTR}>DaI_{PFTR}$.\\
Im Falle autokatalytischer Reaktionen erreicht man den h�chsten Umsatz durch Reihenschaltung von CSTR und PFTR.

\paragraph{Sonderfall: enzymkatalisierte Reaktion}
\bi
\item hyperbolischer Geschwindigkeitsansatz
\item Massenstrom linear mit $X$, Reaktionsgeschwindigkeit nicht linear mit $X$
\item[$\Rightarrow$] mehrere stabile Betriebspunkte im $r$-$X$-Diagramm
\ei

\subsubsection{Enthalpiebilanz f�r CSTR}
\[ \fbx{ \underbrace{ \frac{\partial \left( \rho c_pT \right)}{\partial t} }_{ \mbox{Akk.} } =
\underbrace{-\div \left( \rho c_p T \bar u \right)}_{ \mbox{Konvektion} } + \underbrace{\div \left( \lambda_e \grad T \right)}_{ \mbox{eff. W�rmeltg.} } +
\underbrace{\sum_j \left( -\Delta H_R \right)_j r_j}_{ \mbox{Reaktion} } } \qquad\left[\frac{kJ}{m^3s}\right]  \]
Dimensionslose W�rmebilanz:
\[ \fbx{\underbrace{\left( 1 + H_W \right)}_{\mbox{W�rmetransfer}} \underbrace{ \left( \Theta - \Theta_A \right)}_{\mbox{K�hldifferenz}} = \underbrace{\frac{\tau q_R}{\rho c_p T_0}}_{\mbox{W�rmequelle}}} \]
mit dimensionslosen Kennzahlen:
\begin{eqnarray}
\mbox{W�rmetransferzahl:} & & \quad H_W = \frac{k_w F_k}{\dot V \rho c_p} \nonumber \\
\mbox{(K�hl-) Temperaturerh�hung:} & & \quad \Theta = \frac{T}{T_0}; \quad \left( \Theta_K = \frac{\bar T_K}{T_0} \approx 1,0 \right) \nonumber \\
\mbox{Eingangs- zu K�hltemperatur:} & & \quad \Theta_A = \frac{T_0 + H_W \bar T_K}{\left( 1 + H_W \right) T_0} \nonumber
\end{eqnarray}
Annahme $n=1$ und ersetzen von $X$ durch 
\[ X = \frac{DaI}{1 + DaI}=\frac{\tau k_0 \exp \left( - \gamma / \Theta \right)}{1 + \tau k_0 \exp \left( - \gamma / \Theta \right)} \]
f�hren zu:
\[ \fbx{\underbrace{\left( 1 + H_W \right) \left( \Theta - \Theta_A \right)}_{\mbox{W�rmeabfuhrgerade}} = \underbrace{\frac{\Delta T_{ad}}{T_0} \frac{\tau k_0 \exp \left( - \gamma / \Theta \right)}{1 + \tau k_0 \exp \left( - \gamma / \Theta \right)}}_{\mbox{W�rmeerzeugungskurve}}} \]

\paragraph{2 F�lle}
\be
\item $H_W$ variiert $\Rightarrow$ Steigung der WAG variiert ($\Theta_K = const.$).
\item $\Theta_K$ und somit $\Theta_A$ variiert $\Rightarrow$ WAG wird parallelverschoben ($H_W = const.$).
\item[$\Rightarrow$] Somit ist es m�glich den gew�nschten Betriebspunkt einzustellen.
\ee
\bi
\item {\bf isothermer Fall} $H_W \rightarrow \infty $: Steigung: $\infty$; WAG$=0$: $\Theta = \Theta_K$ 
\item {\bf allgemeiner Fall} Steigung: $1 + H_W$; WAG$=0$: $\Theta = \Theta_A$ 
\item {\bf adiabater Fall} $H_W = 0$: Steigung: $1$; WAG$=0$: $\Theta = \Theta_A = 1$ 
\ei
Polpunkt liegt in ($\Theta_K; \ \Theta_K - 1$); Grenzfall $T_0=\bar T_K \ \Rightarrow \ \Theta_K=1$: Pol($1; \ 0$)\\
\bi
\item[$\Rightarrow$] Vergleiche Z�nd-/L�schverhalten (Kernfachvorlesung)
\item Kriterium f�r stabilen Betriebspunkt:
\ei
\[ \mbox{Steigung}_{WAG} \stackrel{!}{>} \mbox{Steigung}_{WEK} \]
\bi
\item[PFTR:] nur ein Abschnitt befindet sich im kritischen Fall im z�ndwilligen Bereich von $X$.
\item[CSTR:] wenn $X$ in z�ndwilligen Bereich kommt, dann geht der ganze Reaktor durch!
\ei
\subsubsection{Instation�re Betriebsweise}
\bi
\item beim Hochfahren des Reaktors bis zum gew�nschten Betriebspunkt
\item beim �ndern des Betriebszustandes
\item[$\Rightarrow$] Wie lange dauert die instation�re Periode?
\ei
Aus instation�rer und station�rer Massenbilanz folgt f�r die Einstellung von $c_{A,s}$:
\[ t = -\frac{\tau}{ 1 + k \tau} \ln \left( \frac{c_{A,s} - c_A}{c_{A,s}} \right) \]
$\Rightarrow$ wichtiges Kriterium ist hier die Verweilzeit!


