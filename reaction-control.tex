\section{Reaktionsf�hrung}
\subsection{Maximale Produktivit�t in einem Batch-Reaktor}
In einem Batch-Reaktor ($n > 0$) nimmt die Reaktionsgeschwindigkeit mit zunehmer N�he zum Gleichgewichtsumsatz ab.\\
Deswegen stellt sich die Frage nach dem optimalen Umsatz $X_{opt}$ und der optimalen Reaktionszeit $t_{opt}$.
Produktivit�t ergibt sich zu:\\
\begin{minipage}[t]{.4\linewidth}
\[ \fbx{L_p = \frac{c_{10} V X(t)}{t+t_{opt}} } \]
Somit ergibt sich:
\[ \left( \frac{dX}{dt} \right)_{opt} = \frac{X_{opt}}{t+t_{opt}}\]
\end{minipage}
\begin{minipage}[t]{.6\linewidth}
\bild[width=50mm]{images/mincosts.eps}
\end{minipage}
Im Diagramm ist diese leicht grafisch bestimmbar:
\[ \left( \frac{dX}{dt} \right)_{cost} = \frac{X_{cost}}{t + \frac{K_3 + K_2 t_{tot}}{K_1}} \]

\subsection{Minimale Kosten in einem Batch-Reaktor}
Die Kosten f�r den Betrieb eines Batch-Reaktors setzen sich zusammen aus:
\bi
\item $K_1$ zeitabh�ngige Kosten w�hrend des Betriebs
\item $K_2$ zeitabh�ngige Kosten w�hrend der Reinigung
\item $K_3$ Feste Kosten pro Ansatz
\ei

\subsection{Optimale Temperaturf�hrung}
Die optimale Temperaturf�hrung h�ngt von der Art der Reaktion ab:
\bi
\item {\sc Irreversible} Reaktionen: Die h�chste Reaktionsgeschwindigkeit wird bei h�chster Temperatur erreicht. In so fern sollte die Reaktion bei h�chster wirtschaftlich sinnvoller Temperatur ablaufen.
\item {\sc Endotherme} Reaktionen: Durch hohe Temperatur wird die Reaktionsgeschwindigkeit und das Gleichgewicht g�nstig verschoben. Somit auch hier: hohe Temperatur.
\item {\sc Exotherme} Reaktion: Durch niedrige Temperatur wird das Gleichgewicht g�nstig verschoben, aber die Reaktionsgeschwindigkeit nimmt ab. Hier liegt ein Optimierungsproblem vor
\ei
Die optimale Temperatur liegt bei $\frac{\partial R_{Produkt}}{\partial T} = 0$.\\
Zur L�sung f�r ein konkretes Problem geht man von $R$ aus und l�st nach Einsetzen von {\sc Arrhenius} und Einf�hren des Umsatzgrades nach $T$ auf.

\subsection{Komplexe Reaktionen}
Hier bieten sich 3 M�glichkeiten an, um die Selektivit�t zu verbessern:
\be
\item Einsatz eines selektiven Katalysators bzw. L�semittels
\item Temperaturkontrolle (vgl. {\sc Wheeler})
\item Konzentrationskontrolle (vgl. folgende Kapitel)
\ee

\subsubsection{Parallelreaktionen}
Allgemein gilt:
\bi
\item PFTR f�rdert Reaktionen h�herer Ordnung
\item CSTR f�rdert Reaktionen niedrigerer Ordnung
\ei
Die Selektivit�t ergibt sich hier zu:
\[ S = \frac{r_1}{r_2} = \frac{k_1}{k_2} c_1^{n-m} \]
\bi
\item $n > m$ - Hier sollte $c_1$ so hoch wie m�glich sein.
\item $n < m$ - Hier sollte $c_1$ so niedrig wie m�glich sein.
\item $n = m$ - Hier hat die Konzentration keinen Einfluss.
\ei
Sind mehrere Edukte beteiligt wird die Betrachtung schwieriger:
\[ S = \frac{r_1}{r_2} = \frac{k_1}{k_2} c_1^{n_1-m_1} c_2^{n_2-m_2} \]
Hierbei ergibt sich eine Matrix. Kurz gesagt: h�here Ordnung braucht h�here Konzentration.

\subsubsection{Folgereaktion}
F�r eine Folgereaktion wird ebenfalls optimale Ausbeute erw�nscht. Im Vergleich zwischen PFTR und CSTR ergibt sich aus:
\[ X_{PFTR} = 1 - \exp (-DaI) \qquad X_{CSTR} = \frac{DaI}{1+DaI} \]
eingesetzt in $\frac{c_2}{c_{10}}$, dass hier, unabh�ngig von $\frac{k_1}{k_2}$ stets beim PFTR bei geringerer $DaI$ eine gr��ere Selektivit�t vorliegt.
