\section{Sicherheitsaspekte}
\subsection{Unfallursachen}
\begin{minipage}[t]{.45\linewidth}
\bi
\item mechanisches Versagen
\item Bedienungsfehler
\item Ursache unbekannt
\item Prozess-St�rung
\ei
\end{minipage}
\hfill
\begin{minipage}[t]{.45\linewidth}
\bi
\item Naturgewalt
\item Auslegungsfehler
\item Brandstiftung, Sabotage
\ei
\end{minipage}
\paragraph{Beispiel:} stark exotherme Oxidation von o-Xylen im PFTR\\
Partialdruckerh�hung von o-Xylen f�hrt zu:
\be
\item unerw�nschtem Hot-Spot
\item "Durchgehen" des Reaktors
\item[$\lightning$] kleine Schwankungen von Konzentration und Temperatur liegen im Bereich von Dosierungenauigkeiten.
\ee
Kritische Betriebsbereiche vermeiden!\\
$\Rightarrow$ Empfindlichkeitsdiagramm von {\sc Berkelew}:
\bi
\item dimensionsloser adiabater Temperaturanstieg $S$ ($\Delta H_R; \ E_A; \ T_0;...$)
\item dimensionslose W�rmeabfuhrzahl $N$ (W�rmeleitung, Reaktorgeometrie, Reaktion)
\item[$\Rightarrow$] Auftragung von $N/S$ gegen $S$:
\item oberhalb der Kurve: $S$ klein, $N/S$ gro�: unkritischer Betrieb
\item unterhalb der Kurve: $S$ gro�, $N/S$ klein: kritischer Bereich
\item Erh�hung von $T_0$ f�hrt zur Ausdehnung des kritischen Bereiches
\item Erh�hung der Reaktionsordnung f�hrt zur Ausdehnung des sicheren Bereiches
\ei
\subsection{Unfallfolgen}
%\begin{minipage}[t]{.27\linewidth}
%\be
%\item "Durchgehen" von Reaktionen
%\item Brand oder Explosion
%\\
%\item Freiwerden gef�hrlicher Stoffe
%\ee
%\end{minipage}
%\hfill
%\begin{minipage}[t]{.63\linewidth}
%\bi
%\item H�chst, 1993, Nitrochlorbenzol
%\item Flixborough, GB, 1974, Cyclohexan
%\item Oppau, BASF, 1921, Ammoniumnitrat
%\item Seveso, 1976, "Sevesodioxin" TCDD
%\item Bophal, Indien, 1984, Methylisocyanat
%\item Basel, Sandoz, 1986, kontaminiertes L�schwasser im Rhein
%\ei
%\end{minipage}
\bi
\item[1.] {\bf "Durchgehen" von Reaktionen:}
\item H�chst, 1993, Nitrochlorbenzol
\item[2.] {\bf Brand oder Explosion:}
\item Flixborough, GB, 1974, Cyclohexan
\item Oppau, BASF, 1921, Ammoniumnitrat
\item[3.] {\bf Freiwerden gef�hrlicher Stoffe:}
\item Seveso, 1976, "Sevesodioxin" TCDD
\item Bophal, Indien, 1984, Methylisocyanat
\item Basel, Sandoz, 1986, kontaminiertes L�schwasser im Rhein
\ei
\subsection{M�glichkeiten des Gefahrenschutzes}
\be
\item Umhausung, Abschirmung
\item MSR-Technik, Schnellschlussventile
\item eigensichere Prozesse
\ee
Ein Prozess ist eigensicher, wenn keine St�rung zu einem Unfall f�hren kann:
\bi
\item Identifizierung aller m�glichen Reaktionen
\item Minimierung der Menge gef�hrlicher Stoffe
\item Vermeidung aller externer Ausl�ser von "Runaways"
\item Betriebsbedingungen fern vom kritischen Bereich
\ei
\section{Nachhaltige Chemieproduktion}
Nachhaltige Entwicklung erf�llt die Bed�rfnisse unserer Generation auf eine Art und Weise, die auch zuk�nftigen Generationen erm�glicht, ihre Bed�rfnisse zu erf�llen.
\[ \mbox{Nachhaltigkeit = �kologie + �konomie + soziale Aspekte} \]
\subsection{"Gr�ne" chemische Prozesse}
\bi
\item Energie- und Atom-Effizienz
\item Katalysatoren
\item umweltfreundliche L�semittel
\item Prozess�berwachung und Sicherheit
\item alternative Rohstoffe
\item neue Produkte
\ei


