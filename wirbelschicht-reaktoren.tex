\section{Wirbelschicht-Reaktoren}
Bei Wirbelschichten unterscheidet man zwischen 6 Stadien (vgl. {\sc MVT}):
\be
\item $u < u_{mf}$: Festbett
\item $u \approx u_{mf}$: Lockerungszustand
\item $u > u_{mf}$: Blasenbildende Wirbelschicht
\item $u > u_{mf}$: Sto�ende Wirbelschicht
\item $u > u_{mf}$: Kanalbildende Wirbelschicht
\item $u \gg u_{mf}$: Feststoffaustrag - Zirkulierende Wirbelschicht
\ee
Der charakteristische Verlauf des Druckverlusts �ber der Str�mungsgeschwindigkeit erm�glicht die Bestimmung von $u_{mf}$.
�ber die H�he des Bettes am Lockerungspunkt kann $\varepsilon_{mf}$ bestimmt werden: 
\[ \varepsilon_{mf} = \frac{V_{s}}{V_{ges}} = A \frac{H_{\mbox{Festbett}}}{H_{\mbox{Lockerungspunkt}}} \]
Zur Beschreibung einer Wirbelschicht bedient man sich der Kr�ftebilanz:
\[ \fbx{ \Delta p = (1 - \varepsilon_{mf}) H g (\rho_p - \rho_f) } \]
Technisch unterscheidet man zwischen:
\bi
\item {\bf Homogene} Wirbelschicht: gleichm��ig expandiert mit $u < 3 u_{mf}$; Typ 2 und 3 (s.o.); Bsp. katalytische Gas-Phasen-Reaktion
\item {\bf Heterogene} Wirbelschicht: hoch expandiert mit $u > 3 u_{mf}$; Typ 4, 5 und 6; Bsp. Kohleverbrennung oder Rauchgasreinigung.
\ei

\subsection{Homogene Wirbelschichten}
Die wichtigste Eigenschaft der homogenen Wirbelschicht ist die Isothermie. Deswegen kann auf eine Enthalpiebilanz komplett verzichtet werden.
Ausgehend vom eindimensionalen Dispersionsmodell kann nun die Massenbilanz aufgstellt werden:
\[ \fbx{0 = - u \frac{dc_1}{dh} + D_a \frac{d^2c_1}{dh^2} + k c_1 } \]
Hierbei ist zu bemerken, dass $D_a = f(d_p;\Psi)$, also von Partikelgr��e und Beschaffenheit (Spherizit�t) abh�ngig ist.

\subsection{Heterogene Wirbelschichten}
F�r eine heterogen fluidiesierte Wirbelschicht ist es schwer, eine angemessene Beschreibung f�r den Massentransport zwischen Gasblasen und Katalysator-{\emph Suspension} zu finden. Hier muss auf Modelle zur�ckgegriffen werden:

\subsubsection{Blasenmodell}
Dieses Modell wurde von {\sc Kunii} und {\sc Levenspiel} entwickelt und beschreibt den Massentransport zwischen Blase und Suspension in zwei Schritten:
\be
\item Massentransport Blase$\rightarrow$Wolke $K_{bc}$
\item Massentransport Wolke$\rightarrow$Suspension $K_{cs}$
\ee
Man beschreibt eine aufsteigende Blase hierbei als einen Batch-Reaktor mit Reaktion 1. Ordnung:
\[ 0 = - u_b \frac{dc_b}{dh} - K_{eff} c_b \]
Dabei vereinigt $K_{eff}$ alle kinetischen und massentransportbehafteten Effekte.
Zur Bestimmung von $K_{eff}$ trifft man folgende Annahmen:
\bi
\item Jegliches Gas steigt in Form von Blasen durch die (sozusagen autark) fluidisierte Wirbelschicht
\item Alle Gasblasen verhalten sich gleichartig in Bezug auf Verweilzeit, Gr��e und Form
\item Der Gasaustausch erfolgt �ber die Wolke/Schleppe der Blase
\item Stationarit�t
\ei
Hieraus ergibt sich folgendes L�sungskonzept:
\be
\item Aufstellen von Bilanzgleichungen f�r Blase B, Wolke W und Suspension S:
\ee
\begin{eqnarray}
\mbox{Gesamtumsatz} &=& \mbox{Reaktion in B} + \mbox{Transport B-W} \\
\mbox{Transport B-W} &=& \mbox{Reaktion in W} + \mbox{Transport W-S} \\
\mbox{Transport W-S}&=& \mbox{Reaktion in W}
\end{eqnarray}
\bi
%Reaktion in Blase = Transport von Wolke\\
%Transfer von Wolke = Reaktion in Wolke + Transport von Suspension\\
%Transport von Suspension = Reaktion in Suspension
\item[2.] L�sen dieser durch Einsetzen von (3) in (2) und (2) in (1).
\item[3.] Eliminieren der unbekannten Konzentrationen $c_s$ und $c_c$.
\item[4.] Aufl�sen nach $K_{eff}$.
\ei
Nach L�sung der Massenbilanz
\[ (1-X) = \exp \left( - \frac{K_{eff} H}{u_b} \right) \]
kann $1-X$ in Abh�ngigkeit von $H$ grafisch aufgetragen werden.\\
$\ominus$ Der unsicherste Parameter ist hierbei $u_b$, welcher abh�ngig von der Blasengr��enverteilung aufgetragen wird.
\bi
\item experimentelle Ermittlung
\item (nachtr�gliche) Anpassung der Parameter
\item[$\Rightarrow$] kein A-Priori-Modell!
\ei

\subsubsection{Zweiphasenmodell}
In diesem Modell von {\sc Van Deemter} wird angenommen, dass sich das einstr�mende Gas in zwei Teile aufspaltet:
\be
\item Teil zum Fluidisieren des Betts. Str�mt durch Suspensionsphase mit $u_{mf}$.
\item Teil formt Blasen, welche mit $u - u_{mf}$ durch den Reaktor wandern.
\ee
Zwischen Blase und Suspension findet ein Massentranfer $\alpha$ statt.
Eine Reaktion findet ausschlie�lich in der Suspension statt (hier ist der Unterschied zum Blasenmodell!).\\
Massenbilanz f�r Suspensionsphase:
\[ 0 = - u_{mf} \frac{dc_s}{dh} + D_a \varepsilon_{mf} \frac{d^2c_s}{dh^2} + \alpha (c_b -c_s) + (1-\varepsilon_b)(1-\varepsilon_{mf}) \rho_s k' c_s \]
Massenbilanz f�r die Blase-Phase:
\[ 0 = - (u - u_{mf}) \frac{dc_b}{dh} - \alpha (c_b - c_s) \]

\subsubsection{Filmmodell}
{\sc Werther} entwickelte als Erweiterung des Zweiphasenmodells das Filmmodell, welches eine Behandlung analog der Fluid-Fluid-Reak\-tionen vorschl�gt.
\bi
\item Bilanzierung um Blase, Film und Suspension
\item Massentransport durch die Phasengrenze wird einer Adsorption gleichgesetzt
\item pseudo-homogene Reaktion in der Suspensionsphase wird angenommen
\ei
Durch die �blichen Randbedingungen erfolgt eine analytischen L�sung, welche, wie bereits aus der Fluid-Fluid-Beschreibung bekannt, die $Ha$-Zahl und $B$ enth�lt.
\bi
\item[$\oplus$] Zug�nglichkeit von $k_g$, $a$ (spez. Oberfl�che in $B$) und $\varepsilon_{mf}$ �ber Korrelationen
\item[$\Rightarrow$] {\bf A-Priori-Design} von Wirbelschichten ohne Experimente ist m�glich!
\ei
