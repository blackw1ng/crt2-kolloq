\documentclass[9pt,a4paper,fleqn,german]{scrartcl}

% enc
\usepackage[T1]{fontenc}
\usepackage[latin1]{inputenc}

\usepackage{german}
\usepackage[german]{babel}

% tweaking f�r pdf 
\usepackage{ae}
\usepackage{a4}
%\usepackage{anysize}
\usepackage[left=1cm,top=1cm,right=1cm,bottom=1.2cm,nofoot,nohead]{geometry}
\pagestyle{plain}

% references
\usepackage[ps2pdf]{hyperref}
\usepackage{varioref}

% grafiken
\usepackage{graphicx}
%\usepackage{pictex}

% speziell symbole
\usepackage{latexsym}
\usepackage{amsmath}
\usepackage{amssymb}
\usepackage{mathbbol}
\usepackage{pxfonts}
\usepackage{stmaryrd}
\usepackage[small,compact]{titlesec}
\usepackage{float}

% itemstacking
\usepackage{atbeginend}
\AfterBegin{itemize}{ \addtolength{\itemsep}{-1.7ex}}
\AfterBegin{enumerate}{ \addtolength{\itemsep}{-1.7ex}}
\BeforeBegin{tabular}{ \vspace{-2ex} }
\AfterEnd{tabular}{ \vspace{-2ex} }
\BeforeBegin{figure}{ \vspace{-3ex} }
\AfterEnd{figure}{ \vspace{-3ex} }

% PPCHTEX
\usepackage{m-pictex}
\usepackage{m-ch-de}

\hypersetup{%
pdftitle = {CRT Vertiefungs Formelsammlung},
pdfsubject = {Kurzzusammenfassung und Formelsammlung zum Vertiefungsfach "Chemische Reaktionstechnik", CBI, Erlangen},
pdfauthor = {Florian Enzenberger, Sebastian Werner},
pdfcreator = {GhostScript},
pdfproducer = {LaTeX},
pdfstartview = {FitH},
}

% mehr zeuch aufs blatt
%\setlength{\textheight}{25cm}
%\setlength{\topmargin}{-1cm}
%\setlength{\parskip}{\baselineskip}
\addtolength{\parskip}{-1mm}
\setlength{\parindent}{0pt}

% brauchen latex2e
\NeedsTeXFormat{LaTeX2e}

% neue commands
\newcommand{\osum}{\sum \hspace*{-2.4ex}\vspace*{-1ex} \circ\hspace*{2.4ex}\vspace{1ex}}
\newcommand{\bi}{\begin{itemize}}
\newcommand{\ei}{\end{itemize}}
\newcommand{\be}{\begin{enumerate}}
\newcommand{\ee}{\end{enumerate}}
\def \fbx#1 {\fbox{$ \displaystyle #1$}}
\def \bild[#1]#2 {\begin{figure}[H] \includegraphics[#1]{#2} \end{figure}}
\newcommand{\grad}{ \mathrm{grad} \,}
\renewcommand{\div}{ \mathrm{div} \,}
\newcommand{\tot}{ \mathrm{rot} \,}

\begin{document}

%zweispaltig mit �berschrift zentriert
\twocolumn[{\csname @twocolumnfalse\endcsname
{\centerline{ \Huge \sc Formelsammlung Chemische ReaktionsTechnik II}}
{\centerline{ \it Florian Enzenberger \& Sebastian Werner, 2006 } }
{\centerline{ \bf Keinerlei Anspruch auf Vollst�ndigkeit oder Richtigkeit! }}
\vspace{1ex}
}]

\section{Homogene Katalyse}
\subsection{Allgemeines}
Als Basis dient der Vergleich zwischen homogener und heterogener Katalyse aus dem Kernfach:
\begin{table}[H]
\begin{tabular}{l|c|c}
                	& \bf Homogen   & \bf Heterogen \\ \hline
Aktivit�t       	& hoch          & variabel \\
Selektivit�t		& hoch          & variabel\\
Reaktionsbedingungen	& mild          & hart\\
Lebensdauer     	& variabel      & lang\\
Vergiftungsgefahr  	& niedrig       & hoch\\
Diffussionsprobleme     & niedrig       & hoch\\
Kosten f�r Regeneration	& hoch          & null\\
\end{tabular}
\end{table}
Spezielle Aspekte sind hierbei:
\bi
\item Meist �bergangsmetallkomplexe ($Pd, Rh, Ni, Co$) $\rightarrow$ Metallkomplexchemie
\item Viel Auswahl an Liganden, Oxidationszahlen und Koordinationen m�glich
\item Taylor-Made-Catalysators
\item Deswegen aber auch nur geringe Temperaturen m�glich
\ei
\subsection{Katalytischer Kreislauf}
\begin{center}
\bild[width=65mm]{images/catcycle.eps}
\end{center}
Hierbei haben die Teilschritte auch besondere Bezeichnungen
\bi
\item
\begin{minipage}{.5\linewidth}
\bild[scale=0.65]{images/assoziation.eps}
\end{minipage}
\begin{minipage}{.5\linewidth}
$\rightarrow$ Assoziation / Koordination\\
$\leftarrow$ Dissoziation
\end{minipage}

\item 
\begin{minipage}{.5\linewidth}
\bild[scale=0.65]{images/addition.eps}
\end{minipage}
\begin{minipage}{.5\linewidth}
$\rightarrow$ Oxidative Addition \\
$\leftarrow$ Reduktive Elimination
\end{minipage}

\item
\begin{minipage}{.5\linewidth}
\bild[scale=0.65]{images/insertion.eps}
\end{minipage}
\begin{minipage}{.5\linewidth}
$\rightarrow$ Insertion\\
$\leftarrow$ Extrusion
\end{minipage}

\item
\begin{minipage}{.5\linewidth}
\bild[scale=0.65]{images/coupling.eps}
\end{minipage}
\begin{minipage}{.5\linewidth}
$\rightarrow$ Oxidative Coupling\\
$\leftarrow$ Reductive Cleavage
\end{minipage}
\ei

\subsection{Charakterisierung von Katalysatoren}
\subsubsection{Turnover Frequency (TOF)}
\[ \fbx{ TOF = \frac{\mbox{Bildungsgeschwindigkeit}}{\mbox{Katalysatorkonzentration}} = \frac{R_j}{c_{Kat}} } \]
\subsubsection{Turnover Number}
\[ \fbx{ TON = \frac{\mbox{Max. Menge Produkt bis Kat. deaktiviert}}{\mbox{Menge Katalysator}} = \frac{n_{Prod,max}}{n_{Kat}} } \]
\subsubsection{Productivit�t}
�hnlich wie TON, pro Katalysatormenge in bestimmter Zeit umgesetzte Masse 
\[ \fbx{ P = \frac{m_{Prod}}{m_{Kat} \cdot t} } \]
\subsubsection{Selektivit�t}
\[ \fbx{  S_{ki} = \frac{\mbox{Menge an gebildetem Produkt k}}{\mbox{Menge an umgesetztem Reaktanden i}} = \frac{Y_{ki}}{X_i} = \frac{\dot{n}_k - \dot{n}_{k0}}{\dot{n}_{i0} - \dot{n}_i} \frac{\left|\nu_i\right|}{\left|\nu_k\right|} } \]


\section{Fluid-Fluid Reaktionssysteme}
\subsection{Allgemeines}
Man unterscheidet in der Praxis zwischen
\be
\item Gas-Fl�ssig- und
\item Fl�ssig-Fl�ssig-Systeme
\ee
Im Allgemeinen findet die Reaktion in einer der fl�ssigen Phasen statt, in welcher auch der Katalysator gel�st ist. 
Die Aufgabe der anderen Phase ist hierbei, das Extraktionssolvent f�r das Produkt und den optimalen Transport der Edukte zu sein. 
Ein schneller Abtransport der Produkte aus dem Reaktionsmedium ist im Hinblick auf Folgereaktionen besonders wichtig. Hieraus ergeben sich f�r solche Systeme folgende Randbedingungen:
\bi
\item Reaktionsphase muss den Katalysator l�sen und immobilisieren
\item Deaktivierung des Katalysators sollte gering sein
\item L�slichkeit der Edukte in Reaktionsphase sollte hoch sein
\item Mischungsl�cke zwischen Produkt und Reaktionsphase (Direkte Abtrennung des Produkts zur Vermeidung von Folgereaktionen) vorteilhaft
\ei
F�r die Modellierung geht man wie folgt vor:
\bi
\item Massentransfer in beiden Phasen
\item Phasen�bergang
\item Gradient an Phasen�beangsfilm kontrolliert Transportrate
\ei

\subsection{Zweifilm-Theorie}
\begin{minipage}[t]{.65\linewidth}
\bi 
\item Stagnierender Film an beiden Seiten der Grenzfl�che
\item Gesamter Transportwiderstand in diesem Bereich
\item Gleichgewicht an Phasengrenzfl�che
\item Im Film nur molekulare Diffusion ($\rightarrow$ {\sc Fick}'sches Gesetz)
\ei
\end{minipage}
\hfill
\begin{minipage}[t]{.3\linewidth}
\bild[height=25mm]{images/twofilm.eps}
\end{minipage}

Daraus ergibt sich:
\[ \fbx{J_{i,g} = - D_{i,g} \frac{\Delta c_{i,g}}{\delta_g} = - D_{i,l} \frac{\Delta c_{i,l}}{\delta_l} = J_{i,l}} \quad [mol/m^2] \]
Dabei wird oftmals der {\bf Massentransportkoeffizient} $k$ verwendet:
\[ \fbx{ k = \frac{D}{\delta} } \quad [m/s] \]
An den Phasengrenzfl�chen gilt:
\[ \mbox{\sc Henry}\quad p_{i,g}^* = H_i \cdot c_{i,l}^* \]
\[ \mbox{\sc Nernst}\quad c_{i,l}^* = K_{N,ij} \cdot c_{j,l}^* \]
In der Praxis sind die Konzentrationen an der Phasengrenzfl�che (gekennzeichnet durch $^*$) meist unbekannt. Diese k�nnen bei Berechnung durch Gleichsetzen der Flow-Rates $J_{i}$ eliminiert werden.

\subsection{Kopplung von Reaktion und Massentransport}
\subsubsection{Langsame Reaktion - Reaktion im Bulk}

\begin{minipage}[t]{.65\linewidth}
Die Reaktion an Phasengrenzf�che kann vernachl�ssigt werden.
Es wird vorausgesetzt, dass $c_b$ im �berschuss vorhanden ist.
Somit:
\[ \rightarrow J \cdot a = k_l \cdot a \left( c_a^* - c_{a,l} \right ) \]
\end{minipage}
\hfill
\begin{minipage}[t]{.3\linewidth} 
\bild[height=25mm]{images/langsam-imbulk.eps}
\end{minipage}
Unter Verwendung von
\[ r = c_{a,l} \cdot c_{b,l} \cdot k' \quad \mbox{und} \quad c_b = c_{b,l} = const \]
ergibt sich:
\[ r_{eff} = \left[ \frac{1}{k_l a} + \frac{1}{k} \right]^{-1} c_a^* \]
wobei $c_a^ = \frac{p_{a,g}}{H_a}$ gilt.

\subsubsection{Schnelle Reaktion - Reaktion im Film}
\begin{minipage}[t]{.65\linewidth}
Kombination von Massentransport und Reaktion. �hnlich der Situation am Katalysatorkorn. Kann durch Massenbilanz gel�st werden.
Ansatz:
\[ \left.(J_a a)\right|_{y} - \left.(J_a a)\right|_{y+dy} = k' c_a c_b a\, dy \]
\end{minipage}
\hfill
\begin{minipage}[t]{.3\linewidth}
\bild[height=25mm]{images/schnell-imfilm.eps}
\end{minipage}
L�sung durch Anwendung von {\sc Fick}'schem Gesetz und {\sc Taylor}-Entwicklung, sowie der Vereinfachungen $p^* = p_g$ und $c^* = \frac{p_g}{H}$ ergibt L�sung, welche die $Ha$-Zahl enth�lt.\\
\begin{minipage}[t]{.65\linewidth}
\[ \fbx{ Ha = \delta \sqrt{\frac{k\cdot c_1^{* n-1}}{D_{1,l}}} } \]
Das Konzentrationsprofil �ndert sich logischerweise drastisch mit Reaktionsgeschwindigkeit.
\end{minipage}
\hfill
\begin{minipage}[t]{.3\linewidth}
\bild[height=25mm]{images/konz-verlauf.eps}
\end{minipage}

\subsection{Wirkungsgrad der Fl�ssigkeit}
Die Definition des Wirkungsgrades ist 
\[ \fbx{ \eta = \frac{r_{eff}}{r_{max}} } = \frac{Ha}{\tanh Ha} \left( 1 - \frac{c_{1l}}{\cosh Ha} \right) \frac{k_{1l}a}{k} \]
Hierbei sind die Grenzf�lle f�r Fluid-Fluid-Reaktionen interessant.
\begin{minipage}[t]{.45\linewidth}
$k \rightarrow 0$ und $Ha \rightarrow 0$
\[ \eta = \frac{1}{\frac{k}{k_l a} + 1} \]
\end{minipage}
\hfill
\begin{minipage}[t]{.45\linewidth}
$k \gg 1$ und $Ha > 3$
\[ \eta = a \sqrt{\frac{d_l}{k}} \]
\end{minipage}
Folglich kann der Wirkungsgrad durch gr��ere Phasengrenzfl�che verbessert werden.

\subsection{Enhancement Faktor}
Der Enhancement-Faktor beschreibt die Ver�nderung des Stoff�berganges unter Einfluss einer Reaktion:
\[\fbx{ E = \frac{J_{\mbox{mit Reaktion}}}{J_{\mbox{ohne Reaktionseinfluss}}} } \]
Hierbei stellen sich dann 5 Grenzf�lle ein:
\be
\item
\begin{minipage}[t]{.75\linewidth}
Langsame Reaktion $Ha < 0.3$\\
$E = 1$, Massentransfer unbeeinflusst von Reaktion.
Alles reagiert im Bulk.
\end{minipage}
\hfill
\begin{minipage}[t]{.2\linewidth}
\bild[height=20mm]{images/enhance-1.eps}
\end{minipage}

\item
\begin{minipage}[t]{.75\linewidth}
Mittelschnelle Reaktion $0.3 < Ha < 3$\\
$E > 1$, Reaktion beschleunigt Massentransport.
Reaktion teilweise im Film.
\end{minipage}
\hfill
\begin{minipage}[t]{.2\linewidth}
\bild[height=20mm]{images/enhance-2.eps}
\end{minipage}

\item
\begin{minipage}[t]{.75\linewidth}
Schnelle Reaktion $Ha > 3$ \\
$E \approx Ha$, Massentransport unabh�ngig von $k_l$!
Hier: $J = c^* \sqrt{k \cdot D_l}$; Reaktion vollst�ndig im Film.
Pseudo-Erster-Ordnung!
\end{minipage}
\hfill
\begin{minipage}[t]{.2\linewidth}
\bild[height=20mm]{images/enhance-3.eps}
\end{minipage}

\item
\begin{minipage}[t]{.75\linewidth}
$Ha \gg 3 \quad - \quad 5 E_{max} > Ha > \frac{E_{max}}{5}$\\
Sehr schnelle Reaktion bereits im Diffusionslayer.
Vergleichbar Reaktion zweiter Ordnung!
\end{minipage}
\hfill
\begin{minipage}[t]{.2\linewidth}
\bild[height=20mm]{images/enhance-4.eps}
\end{minipage}

\item
\begin{minipage}[t]{.75\linewidth}
$E_{max} < \frac{Ha}{5}$
Sofortige Reaktion. Reaktion direkt an Phasengrenzfl�che.
Beispiel: S�ure + Lauge.
\end{minipage}
\hfill
\begin{minipage}[t]{.2\linewidth}
\bild[height=20mm]{images/enhance-5.eps}
\end{minipage}
\ee

Im nachfolgenden Graph sind die 5 F�lle erkennbar:
\bild[width=80mm]{images/e-ha.eps}

Das Maximal m�gliche Enhancement ergibt sich zu:
\[ \fbx{E_{max} = 1 + \frac{\nu_1}{\nu_2} \frac{D_2}{D_1} \frac{c_{2,l}}{c^*_1}} \]
Im Falle maximalen Enhancements kommt die Reaktion quasi zum Reaktanden.

\subsection{Bestimmung der optimalen Austauschfl�che a}
F�r eine optimale Reaktion muss die Massentransporthemmung minimal sein.
F�r die Auswahl eines Reaktortyps, welcher entsprechenden Massentransport und Verweilzeiten erm�glicht, wird �ber die
Kennzahl $B$ eine Beurteilung vorgenommen.
\[ \fbx{ B = \frac{\mbox{Grenzfilmvolumen}}{\mbox{Fl�ssigkeitsvolumen}} = \frac{A\delta}{V_l} = a \cdot \delta } \]
\begin{figure}[H]
\beginpicture
\setcoordinatesystem units <1.7cm,1.7cm>
\setplotarea x from -4 to 0, y from -2 to 0
\axis bottom label {$\log B$} ticks numbered from -4 to 0 by 1 unlabeled short from -4 to 0 by 0.1 /
\axis left label {\rotatebox{90}{$\log \eta$}} ticks numbered from -2 to 0 by 1 unlabeled short from -2 to 0 by 0.1  /
\axis top /
\axis right /
\setdots <2pt>
\linethickness 1pt
\setquadratic
\setsolid
\plot -4 -0.3 -3 -0.06 0 0 /
\plot -4 -1.3 -2.65 -0.3 0 0 /
\plot -4 -2 -2.5 -0.5 0 0 /
\plot -3 -2 -1.8 -0.7 0 0 /
\plot -2.1 -2 -1 -0.8 0 -0.15 /
\setlinear
\plot -1.5 -2 0 -0.5 /
\put {Ha = 0,01} at -3.5 -0.35
\put {0,05} at -3.75 -0.7
\put {0,1} at -3.1 -0.7
\put {0,3} at -2 -0.7
\put {1,0} at -1.15 -0.7
\put {3,0} at -0.2 -1
\put {operation range} at -1.9 -1.3
\linethickness 3pt
\plot -3.7 -1.4 -1.1 -1.4 /
\endpicture
\end{figure}
�bliches Einsatzgebiet ist: $2\cdot10^{-4} < B < 10^{-1}$.\\
Im Diagramm ist oben rechts der Massentransport ausreichend schnell, w�hrend unten links eine massive Hemmung stattfindet.\\
Folglich:
\bi
\item F�r kleine Ha-Zahlen sorgt ein gr��eres B (Oberfl�che!) nicht f�r gr��ere Kapazit�ten (Reaktion im Bulk!)
\item F�r schnelle Reaktionen (gro�e Ha) muss eine gro�e Oberfl�che (B!) bereitgestellt werden, da sonst $\eta$ sehr klein wird.
\ei

\subsection{Ruhrchemie-Rhone-Poulenc Prozess}
Der RCH/RP-Prozess ist eine technische Umsetzung der Hydroformylierung von Propen. Hierbei sollen bevorzugt die linearen $n$-Aldehyde und nicht die verzweigten $iso$-Aldehyde gebildet werden, da hier ein gr��erer Bedarf besteht. $n$-Aldehyde bieten Zugang zu Poly-Olen, Aminen und Carboxyl-Funktionen.

\subsubsection{Technische Umsetzung}
Die Reaktion wird in einem Biphasischen liquid-Liquid-System mit wassergel�stem Rh-Katalysator (TPPTS) durchgef�hrt, welcher die geforderte Selektivit�t bietet. Das Hauptproblem ist wieder die Abtrennung des Katalysators. Hierf�r bieten sich 3 L�sungen an:
\be
\item Produktabtrennung durch {\bf Strippen}. Hier wird ein gro�er Recycle-Gas-Strom verwendet, wobei der Katalysator im Reaktor verbleibt.
\item Produkte aus Fl�ssigphase {\bf abdestillieren}. Katalysator wird in Hochsieder gel�st und aus Kolonnensumpf zur�ckgef�hrt. Vorteil ist, dass weniger Katalysator ben�tigt wird.
\item Produktabtrennung via {\sc Absetzen}. Hier ist viel, m�glichst hydrophiler Katalysator und ein gro�er Reaktor erforderlich.
\ee

\subsubsection{Reaktorkonzepte}
Das heute typische Konzept ist ein CSTR mit wassergel�stem TPPTS-Katalysator. Interessant ist hierbei, dass das $n$-Aldehyd zum K�hlen des CSTR verwendet wird. Die R�ckf�hrung der w�ssrigen Reaktionsphase wird via Settler erreicht. Die Aldehyd-Phase wird dann, zwecks R�ckf�hrung des unverbrauchten Propens mit frischem Syngas gestripped. Das Roh-Aldehyd wird dann in $n$- und $iso$-Fraktion rektifiziert.\\
Eine aktuelle Variante ist einen PFTR mit turbulenzbildenden Einbauten (z.B. Sulzer SMV). Hier verwendet man ein hochverd�nntes Reaktionsgemisch mit sehr hoher Katalysatorkonzentration. 
Durch die hohe Turbulenz entstehen kleinste Syngasbl�schen und Olefintr�pfchen, was in einer sehr hohen Interface-Area $a$ resultiert. 
In Folge dessen kann ein $STY$ erreicht werden, welches bis zu 10x h�her als beim klassischen Reaktorkonzept ist. 
Untersuchungen haben ergeben, dass hier die Reaktion vor allem im Bulk stattfindet.

\subsection{Spezielle L�sungsmittelkonzepte}
F�r besondere Anwendungen im Bereich der biphasischen homogenenen Katalyse haben sich spezielle L�sungsmittelkonzepte entwickelt.
\subsubsection{Flourierte Phasen}
Verwendung hochflourierter Alkane als Reaktionsphase.
\bi
\item[$\oplus$] Mischungsl�cke mit Wasser und Kohlenwasserstoffen
\item[$\oplus$] Hohe Dichte
\item[$\oplus$] Hohe L�slichkeit von O$_2$
\item[$\ominus$] Hohe Fl�chtigkeit
\item[$\ominus$] Teuer
\item[$\ominus$] Gefahr der Verschmutzung der Produkte
\item[$\ominus$] Hoher Trennaufwand
\ei

\subsubsection{�berkritisches CO$_2$ (scCO$_2$)}
CO$_2$ im �berkritischen Zustand ($T_C = 31.1\,�C$, $P_C = 73.6\,bar$) vereint viele Vorteile von Fl�ssigkeiten und Gasen.
\bi
\item[$\oplus$] Gasartige, geringe Viskosit�t
\item[$\oplus$] Hohes $D_e$
\item[$\oplus$] Geringe Oberfl�chenspannung 
\item[$\oplus$] Dichte wie eine Fl�ssigkeit
\item[$\oplus$] L�sungsmittel f�r s/l-Systeme
\item[$\oplus$] W�rmetransport in der Gr��enordnung von Fl�ssigkeiten
\item[$\oplus$] Preiswert, Ungiftig, Unbrennbar
\item[$\oplus$] Trennaufwand gering (Flash!)
\item[$\ominus$] Metallkomplexe nur gering l�slich
\item[$\ominus$] Hoher Druck notwendig
\ei

\subsubsection{Ionische Fl�ssgkeiten (IL)}
Salze, die unter $100\,�C$ fl�ssig sind, werden ILs genannt.
\bi
\item[$\oplus$] Kein messbarer Dampfdruck $\rightarrow$ Trennaufwand gering
\item[$\oplus$] Hohe Stabilit�t
\item[$\oplus$] Mischungsl�cken mit unpolaren Fluiden
\item[$\ominus$] Teuer
\item[$\ominus$] Giftigkeit ungekl�rt
\item[$\ominus$] Hohe Viskosit�t
\ei
Als Anwendungen bieten sich somit an:
\bi
\item Elektrochemie
\item Analytik
\item Katalyse
\item Solvent
\item Engineering Fluid
\ei

\section{Heterogene Katalyse}
\subsection{Oberfl�che des Katalysatorkorns}
Wie in der homogenen Katalyse ist eine spezielle {\it geometrische} und {\it elektrochemische} Form der Oberfl�che notwendig, um gute Reaktivit�t und Selektivit�t zu erzeugen.

\subsubsection{Ensemble-Effekt}
Ein \emph{Ensemble} ist die Anzahl an Zentren, die f�r eine optimale Reaktion ben�tigt werden. Man unterscheidet zwischen:
\bi
\item {\bf On-Top} Ein Molek�l adsorbiert an einem aktiven Zentrum
\item {\bf Bridge-Bonded} Ein Molek�l ben�tigt zwei benachbarte Zentren
\item {\bf Hollow-Site} Ein Molek�l adsorbiert zwischen drei Zentren
\ei

\subsubsection{Liganden-Effekt}
Man spricht von \emph{Ligand-Effect}, wenn die Eigenschaften eines Ober\-fl�chen-Atoms durch Austauschen eines gleichartigen Nachbar-Atoms durch ein Andersartiges ver�ndert werden.

\subsubsection{Einflu� von Oberfl�chendefekten}
\bi
\item Stufen und Ecken sind aktiver
\item Adatoms (einzelnes Atom auf einer Fl�che) sind besonders aktiv
\item Zusammenfassend passiert an Unregelm��igkeiten am meisten.
\ei

\subsubsection{Zukunft}
Derzeit unterscheiden sich die Oberfl�chenforschung und die Katalysatorenforschung stark. 
W�hrend bei der Oberfl�chenforschung auf ideale Bedingungen, atomare Betrachtungsweise und Erforschung von Fundamentalmechanismen fokussiert wird, wird bei der heterogenen Katalyse stets mit irealen Bedingungen, System-Betrachtungsweise und dem Fokus auf Aktivit�t, Selektivit�t, Stabilit�t und Effektivkinetik gearbeitet. Das Ziel hierbei ist eindeutig die L�cke zu schlie�en.\\
In der (aufsteigenden) Komplexit�t ist zwischen chemischer
\bi
\item Ad-/Desorption
\item Simple Reaktion
\item Selektivit�t
\ei
und Strukturkomplexit�t
\bi
\item Monokristallin
\item Polykristallin
\item Modellkatalysatoren
\item Realit�t
\ei
zu unterscheiden.

\subsubsection{Methoden zur Strukturanalyse}
Hier kann zwischen Volumenmethoden
\bi
\item Na�chemische/UV: Anzahl / Art der Zentren, Oxidationszahl
\item Single Pellet Apparatus: $D_E$
\item Conductometric Detrimination: $\lambda_E$
\item NMR / ESR: Bindungart
\item R�ntgendiffraktometrie: Kristallstruktur
\ei
und Oberfl�chenmethoden
\bi
\item IR-Spektroskopie: Oberfl�chenbeschaffenheit
\item Temperatur Adsorption: Anzahl / Art aktiver Zentren
\item BET / Porosimetrie: Spezifische Oberfl�che, Porengr��enverteilung
\item Secondary Ion Mass Spectroscopy (SIMS): Elementverteilung
\item XPS: Bindungsart
\ei

\subsubsection{Trends}
Die Kombination von molekular definierten Zentren mit heterogenem Support stellt einen aktuellen Trend dar. Hier kann eine erh�hte Selektivit�t durch die spezifische Geometrie des Supports erreicht werden.

\subsection{Porendiffusion}
F�r die Betrachtung des Einflusses der Porendiffusion auf den Massentransport im Pellet wird, wie gewohnt zur Betrachtungsweise via Massenbilanz gegriffen. Hierbei wird vereinfachend angenommen:
\bi
\item Keine Filmdiffusion
\item Isothermie
\item Gleichartige Porenstruktur
\item Spherisches Katalysatorkorn
\item Pseudo-Homogenes System
\item Simple Reaktion
\ei
Ausgehend von der Massenbilanz:
\[ \eta = \frac{4 \pi R^2 D_e \frac{c_s}{R} \left( \frac{df}{dx} \right)_{x=1} }{\frac{4}{3} \pi R^3 k c_s }  \quad \mbox{mit} \quad \left( \frac{df}{dx} \right)_{x=1} = \frac{\phi}{\tanh \phi} - 1\]
ergibt sich
\[ \fbx{\eta = \frac{3}{\phi} \left( \frac{1}{\tanh \phi} - \frac{1}{\phi} \right) } \]
Wie gewohnt: F�r $\phi > 3$ ist $\tanh \approx 1$. Somit: $\eta = \frac{3}{\phi}$.

\subsection{Einfluss der Porendiffusion auf die Selektivit�t}
{\sc Wheeler} entwickelte eine Betrachtungsweise zur Beurteilung der Selektivit�t von Reaktionen.

\subsubsection{Unabh�ngige Reaktionen}
F�r zwei unabh�ngige Reaktionen erster Ordnung der Reaktanden $A$ und $B$ gilt die Selektivit�t $ \sigma$:
\[ \sigma= \frac{r_B}{r_A}= \frac{c_Bk_2 \eta_B}{c_Ak_1 \eta_A} \stackrel{ \phi>3}= \sqrt{ \frac{k_2D_{B,obs}}{k_1D_{A,obs}}} \]
\be
\item gr��eres Molek�l $B$ ist reaktiver
\[ \frac{D_{B,obs}}{D_{A,obs}}<1 \mbox{ und } \frac{k_2}{k_1}>1 \]
\item[$\Rightarrow$] Selektivit�t sinkt, wenn $ \eta$ sinkt
\item reaktiveres Molek�l weist auch wesentlich h�heren $D_{obs}$ auf:
\[ \sqrt{ \frac{k_2D_{B,obs}}{k_1D_{A,obs}}}> \frac{k_2}{k_1} \]
\item[$\Rightarrow$] Selektivit�t erh�ht sich mit sinkendem Wirkungsgrad;\\
Technische Anwendung: formselektive Zeolithe\\
($D_{linear}>D_{verzweigt}$ und $D_{p-Aromaten}>D_{o-Aromaten}$)
\ee

\subsubsection{Parallelreaktionen}
Bei einer Parallelreaktion $A^{ \nearrow \displaystyle \mbox{B}}_{ \searrow \displaystyle \hbox{C}}$ mit gleichen Reaktionsordnungen beeinflusst $ \eta$ nicht die Selektivit�t $S= \frac{k_1}{k_2}$.
Wenn $A \rightarrow B$ eine Reaktion 1. Ordnung und $A \rightarrow C$ eine Reaktion 2. Ordnung ist, steigt die Selektivit�t f�r $B$, wenn $ \eta$ sinkt. $ \Rightarrow$ Manchmal ist ein niedriger Wirkungsgrad zu Gunsten h�herer Selektivit�t erw�nscht.

\subsubsection{Folgereaktionen}
F�r eine Folgereaktion $A \stackrel{k_1} \rightarrow B \stackrel{k_2} \rightarrow C$ gilt: \\
mit Porendiffusion:
\[ \sigma= \frac{r_B}{r_A}= \frac{k_1c_A-k_2c_B}{k_1c_A}=1- \frac{c_B}{c_A S} \quad \mbox{mit} \quad S= \frac{k_1}{k_2} \]
ohne Porendiffusion:
\[ \sigma= \frac{r_B}{r_A}= \frac{ \sqrt{S}}{1+ \sqrt{S}}- \frac{c_B}{c_A \sqrt{S}} \quad \mbox{mit} \quad \phi>3 \]

\subsection{Zeolithe}
\begin{itemize}
\item kristalline, hydratisierte Aluminosilikate mit Netzwerkstruktur
\item 3-dimensionales, polyanionisches Netzwerk
\item bestehend aus SiO$_4$- und AlO$_4$-Tetraedern
\item empirische Formel: M$_{2/n} \cdot$xAl$_2$O$_3 \cdot$ySiO$_2 \cdot$zH$_2$O ($n$ = Valenzelektronen des Kations; $y \ge 2$ = Anzahl gebundener Wassermolek�le
\item regelm��ige Kan�le oder verbundene Zwischenr�ume (Porendurchmesser $<2 \,nm$)
\item Poren enthalten Wassermolek�le oder ladungsausgleichende Kationen
\item Kationen: mobile, austauschbare Alkali- oder Erdalkalimetallionen 
\end{itemize}
Diffusion in Zeolithen liegt im Bereich der konfigurellen (Gr��e von Kohlenwasserstoffen $ \approx$ Kanaldurchmesser) und {\sc Knudsen}-Diffusion (Porenwand beeinflusst Molek�le). \\
Arten der Formselektivit�t:
\begin{itemize}
\item Reaktanden-
\item Produkt-
\item Zwischenformselektivit�t
\end{itemize}
Technisch wichtigste Anwendung f�r Zeolithe: Fluid Ca\-ta\-ly\-tic Crac\-king (FCC)
\begin{itemize}
\item Riser-Downer(=Regenerator)-Prinzip
\item saurer Zeolith-Kat
\item endotherm
\item Druck etwas �ber $1 \,bar$
\item Temperatur $ \approx 500 \,�C$
\end{itemize}

\subsection{Einfluss der Filmdiffusion}
\[ r_{obs}= \beta_A \cdot \left(c_{Ab}-c_{As} \right)=k_s \cdot c^n_{As} \quad \left[ \frac{mol}{m^2 \cdot s} \right] \] 
b: bulk, s: surface
\begin{table}[H]
\begin{tabular}{p{4.3cm}p{4.3cm}}
\bf Starke Limitierung durch Filmdiffusion	& \bf Keine Limitierung durch Filmdiffusion	 \\
(Diffusionsregime)				& (kinetisches Regime)				 \\
$ \rightarrow$ steiler Konzentrationsgradient	& $ \rightarrow$ kein Konzentrationsgradient	 \\
$k_s \gg \beta$					& $ \beta \gg k_s$				 \\	
$r_{obs}= \beta \cdot c_b$			& $r_{obs}=k \cdot c^{n_{true}}_b$		 \\
$n_{obs}=1$					& $n_{obs}=n_{true}$				 \\
$E_{A,obs}<5 \,kJ/mol$				& $E_{A,obs}=E_{A,true}$			 \\
\end{tabular}
\end{table}
Technische Beispiele mit starker Beeinflussung durch Filmdiffusion
\begin{itemize}
\item nicht por�se Katalysatoren, "Shell"-Katalysatoren
\item Katalysatornetze (NH$_3$-Oxidation)
\item katalytische Wandreaktoren (Mikroreaktionstechnik)
\end{itemize}
Ammoniakoxidation
\[ \quad \chemie{NH_3} \chemie{PLUS} \chemie{5O_2} \chemie{EQUILIBRIUM}{Pt-Netz}{ca. 900�C} \chemie{4NO} \chemie{PLUS} \chemie{6H_2O} \qquad \Delta H_R = - 906 \frac{kJ}{mol} \]
\begin{itemize}
\item sehr stark exotherm $ \Rightarrow$ Kontaktzeit: $1/1000 \,s$ (!)
\item Grund f�r nichtpor�ses Pt-Netz: Vermeidung von Folgereaktionen (NO ist instabil)
\end{itemize}
Filmtheorie: Molekulare Diffusion durch einen laminaren Grenzfilm der Dicke $ \delta=f \left(u, \rho, \nu,d_p \right)$.\\
{\sc Fick}'sches Gesetz:
\[ \fbx{ \displaystyle J=-D \cdot \frac{dc}{dx}= \frac{D}{ \delta} \cdot \left(c_b-c_s \right) } \quad \left[ \frac{mol}{m^2 \cdot s} \right] \]
Korrelation unter Verwendung der dimensionslosen Kennzahlen $Sh$, $Re$ und $Sc$ f�hrt zu:
\[ \beta \propto \sqrt{ \frac{u}{d_p}} \] 
Experimentelle Methode zur �berpr�fung des Porendiffusionseinflusses:
\begin{enumerate}
\item Variation der Str�mungsgeschwindigkeit $u$ bei konstanter Verweilzeit
\item Bestimmung des Umsatzgrades $X$
\item Auftragung von $X$ gegen $u$ $ \Rightarrow$ Umsatz erreicht Plateau
\end{enumerate}

\subsection{Gleichzeitiger innerer und �u�erer Massentransport}
Massentransportlimitierung beginnt in der Pore und endet im Film. $\Leftrightarrow$ W�rmetransport umgekehrt.
\[ \eta = f \Bigg( \underbrace{Bi_m = \frac{ \beta \cdot R}{D_e} = \frac{ \mbox{Filmdiffusion}}{ \mbox{Porendiffusion}}}_{ \mbox{{ \sc Biot}-Zahl}} ;\quad \underbrace{ \phi = R \sqrt{ \frac{ k(T_b) \cdot c^{n-1}_b}{D_e} }}_{ \mbox{{ \sc Thiele}-Modul}} \Bigg) \]
F�r $Bi>100$ kann der Einfluss der Fimdiffussion vernachl�ssigt werden.
\subsection{Gleichzeitiger innerer und �u�erer W�rme- und Massentransport}
\[ r_{obs} \left(T_b,c_{Ab} \right)= \eta \cdot r_{true} \left(T_b,c_{Ab} \right)= \eta \cdot k \left(T_b \right) \cdot c^n_{Ab} \]
Bilanz �ber eine Kugelschale der Dicke $d \hat x$:
\[ \left( \mbox{Inflow} \right)_{ \hat x+d \hat x}- \left( \mbox{Outflow} \right)_{ \hat x}= \mbox{Reaktion} \]
\paragraph{Massentransport:}
\begin{eqnarray}
\mbox{Inflow:} \quad&&-4 \pi \cdot \left( \hat x+d \hat x \right)^2 \cdot D_e \cdot \left. \frac{dc}{d \hat x} \right|_{ \hat x+d \hat x} \nonumber \\
\mbox{Outflow:} \quad&&-4 \pi \cdot \hat x^2 \cdot D_e \cdot \left. \frac{dc}{d \hat x} \right|_{ \hat x} \nonumber \\
\mbox{Reaktion:} \quad&&r \cdot dV=k \cdot c^n \cdot4 \pi \cdot \hat x^2d \hat x \nonumber
\end{eqnarray}
\[ \frac{d^2f}{dx^2}+ \frac{2}{x} \frac{df}{dx}= \phi^2 \cdot f^n \cdot \exp \left( \gamma \left(1- \frac{1}{ \Theta} \right) \right) \]
\paragraph{W�rmetransport:}
\begin{eqnarray}
\mbox{Inflow:} \quad&&-4 \pi \cdot \left( \hat x+d \hat x \right)^2 \cdot \lambda_e \cdot \left. \frac{dc}{d \hat x} \right|_{ \hat x+d \hat x} \nonumber \\
\mbox{Outflow:} \quad&&-4 \pi \cdot \hat x^2 \cdot \lambda_e \cdot \left. \frac{dc}{d \hat x} \right|_{ \hat x} \nonumber \\
\mbox{Reaktion:} \quad&&r \cdot \left(- \Delta H_R \right) \cdot dV=k \cdot c^n \cdot \left(- \Delta H_R \right) \cdot4 \pi \cdot \hat x^2d \hat x \nonumber
\end{eqnarray}
\[ \frac{d^2 \Theta}{dx^2}+ \frac{2}{x} \frac{d \Theta}{dx}= \beta_{Pr} \cdot \phi^2 \cdot f^n \cdot \exp \left( \gamma \left(1- \frac{1}{ \Theta} \right) \right) \]
mit: $ \displaystyle \quad f= \frac{c}{c_b} \quad \Theta = \frac{T}{T_b} \quad x= \frac{ \hat x}{R}$ \\
Randbedingungen:
\be
\item $x=1$: $f=1$, $ \Theta=1$  
\item $x=0$: $ \frac{df}{dx}=0$, $ \frac{d \Theta}{dx}=0$
\ee
Dimensionslose Kennzahlen:
\begin{eqnarray}
\mbox{{ \sc Thiele}-Modul:} \quad && \fbx{ \phi=R \sqrt{ \frac{ k( T_b ) \cdot c^{n-1}_b}{D_e} } } \nonumber \\
\mbox{{ \sc Arrhenius}-Zahl:} \quad && \fbx{ \gamma = \frac{E_A}{R \cdot T_b} } \nonumber \\
\mbox{{ \sc Prater}-Zahl:} \quad && \fbx{ \beta_{Pr} = \frac{c_b \cdot D_e \cdot(- \Delta H_R)}{ \lambda_e \cdot T_b} = \frac{\Delta T_{max}}{T_b} } \nonumber
\end{eqnarray}
\bi
\item[$\beta_{Pr}$] �ber Bulktemperatur normalisierte Maximaltemperatur im Pellet; $\beta_{Pr}>0$: exotherm; $\beta_{Pr}<0$: endotherm; aber: i.d. Praxis $�$ selten $>0,1$ 
\item[$\gamma$] Ma� f�r Temperaturabh�ngigkeit der Reaktion
\item[$\eta>1$] f�r kleine Werte von $\phi$, da Anstieg von $k$ den langsamen Konzentrationsabfall �bertrifft
\item[$\eta \gg 1$] sehr stark exotherme Reaktionen ($\eta$-�berhang), i.d. Praxis unerw�nscht $\Rightarrow$ Kat-Deaktivierung
\item[$\beta_{Pr}<0$] $\Rightarrow$ $\eta<0$ f�r alle Werte von $\phi$, da $k$ und $c$ zur Pelletmitte hin sinken
\ei 

\subsection{"A priori"-Kriterien}
Keine Transporthemmung (willk�rliche Festlegung):
\[ \eta=\frac{r_{obs}}{r\left(T_b,\,c_b\right)}=1\pm 0,05 \]
keine Hemmung, wenn...
\subsubsection{Externe W�rmetransportlimitierung (W�rme�bergang)}
{\sc Mears}-Kriterium:
\[ \quad \fbx{ \frac {\left| \Delta H_R \right| \cdot r_{obs} \cdot R}{h \cdot T_b}<0,15 \frac{R \cdot T_b}{E_A}= \frac{0,15}{\gamma} } \]
$h$: W�rmetransportkoeffizient (auch $k \quad \left[ \frac{W}{m^2 \cdot K} \right]$)
\subsubsection{Interne W�rmetransportlimitierung (W�rmeleitung)}
{\sc Anderson}-Kriterium:
\[ \quad \fbx{ \frac {\left| \Delta H_R \right| \cdot r_{obs} \cdot R^2}{\lambda \cdot T_s}<0,75 \frac{R \cdot T_b}{E_A}= \frac{0,75}{\gamma} } \]
$\lambda$: W�rmeleitf�higkeit
\subsubsection{Externe Massentransportlimitierung (Filmdiffusion)}
{\sc Mears}-Kriterium:
\[ \quad \fbx{ \frac {r_{obs} \cdot R}{c_b \cdot \beta} < \frac{0,15}{n} } \]
$n$: Reaktionsordnung
\subsubsection{Interne Massentransportlimitierung (Porendiffusion)}
{\sc Weisz-Prater}-Kriterium:
\[\quad \fbx{ \Phi = \frac{r_{obs} \cdot R^2}{c_s \cdot D_e}=\phi^2 \cdot \eta } \]
$\Phi<6$ f�r Reaktion 0. Ordnung\\
$\Phi<1$ f�r Reakttion 1. Ordnung\\
$\Phi<0,3$ f�r Reaktion 2. Ordnung

\subsection{Einzelpelletreaktor}
Zur experimentellen Bestimmung von $k$ und $D_e$, bestehend aus:
\bi
\item gradientenfreiem, vollst�ndig r�ckvermischtem Schlaufenreaktor
\item linker Seite der Reaktionskammer (Konzentration $c_i$)
\item einzelnem Katalysatorpellet der L�nge $L$ (Koordinate $z'$) und Masse $m_{Kat}$
\item rechter Seite der Reaktionskammer (Konzentration $c_{iB}$)  
\item[$\Rightarrow$] $c_{iB}$ entspricht der Konzentration in der Mitte eines Katalysatorpellets der doppelten Gr��e
\ei
Die Massenbilanz kann ausgehend vom allgemeinen Fall (keine Konvektion, Diffusion �ber ebene Platte $A$ zwischen $z'$ und $z' +dz'$, isotherm, nur Porendiffusion) durch {\sc Taylor}-Entwicklung und Einf�gen der dimensionslosen Gr��en $f$ und $z$ vereinfacht werden:
\[ \fbx{\frac{d^2f}{dz^2} - \phi^2 f^n = 0} \quad \mbox{mit} \mbox \quad f = \frac{c_i}{c_{is}} \quad \mbox{und} \quad z = \frac{z'}{L} \]
\be
\item {\sc Thiele}-Modul $\phi$ aus den Messwerten: $\displaystyle f = \frac{c}{c_s} = \frac{1}{\cosh \phi} $ %\quad \leadsto \phi = \mbox{arcosh} \left( \frac{c_s}{c} \right) $
\item Wirkungsgrad $\eta$ via $\displaystyle \eta = \frac{\tanh \phi}{\phi} $
\item $k$ via $\displaystyle \eta= \frac{r_{eff}}{k \cdot c_{is}} \quad \mbox{mit} \quad r_{eff}=\frac{\dot V \left( c_i-c_{i0} \right)}{\nu_i \cdot m_{cat}} $
\item $D_e$ mit bekanntem $k$ via $\displaystyle \phi=L \sqrt{ \frac{ k \cdot c^{n-1}_s}{D_e} } $
\ee


\section{Feststoff-Gasphasen Reaktionen}
(Nicht katalytische) Reaktionen zwischen Feststoff und Gasphase kommen in der Praxis oftmals vor:
\bi
\item R�sten von sulfidischen Erzen (mit Sauerstoff)
\item Reduktion von Eisenerz (mit Kohlenmonoxid)
\item Vergasung von Kohle zu Syngas
\item Absorption von SO$_2$ durch Kalk
\item Brennen von Kalkstein zu gebranntem Kalk
\ei
\subsection{Por�ser Feststoff (CVI)}
Durch gr��ere spezifische Oberfl�che wird die Reaktionszone gr��er und es kommt zu Massentransportph�nomenen wie bereits bekannt. Die Konzentrationsgradienten h�ngen hierbei direkt mit Reaktionsrate und Diffusion zusammen.\\
Bei Reaktionen ist zwischen 2 F�llen zu unterscheiden:
\be
\item Kleine Poren, Gro�es Korn: Gradient des Edukts zur Kornmitte hin, diffusionsgehemmt.
\item Gro�e Poren, Kleines Korn: Quasihomogene Eduktkonzentration im Korn, Reaktion ist limitierend.
\ee
In diesem Fall spricht man von Chemical Vapor Infiltration ({\bf CVI}).\\
Die Reaktion l�uft in folgenden Schritten ab:
\be
\item Reaktion in der Gasphase von Precursor zum Edukt
\item Filmdiffusion
\item Porendiffusion
\item Oberfl�chenreaktion mit Filmbildung
\ee
R�ckweg analog.

\subsection{Nicht-Por�ser Feststoff}
Durch definierte Reaktionsoberfl�che wandert die Reaktionszone von au�en nach innen durch den Feststoff.\\
Bei der Reaktion ist zwischen 2 F�llen zu unterscheiden:
\be
\item Reaktion bildet ausschliesslich gasf�rmige Produkte: Radius nimmt mit der Zeit ab
\item Reaktion bildet ebenfalls festes Produkt: Das entstehende Produkt \emph{muss} por�s sein. Hier muss die Diffusion durch diesen Bereich betrachtet werden. Hierzu bietet sich das \emph{Shrinking Core Model} (SCM) an.
\ee
Im Fall 2 spricht man von Chemical Vapor Deposition ({\bf CVD}).\\
Die wichtigsten Schritte sind:
\be
\item Konvektiver Transport des Edukts
\item Massentransport zum Substrat (Grenzfilm)
\item Adsorption (aus dem Film)
\item Oberfl�chendiffusion, Reaktion und Bildung des por�sen Produktes
\ee
R�ckweg analog. Der geschwindigkeitsbestimmende Schritt ergibt sich hierbei durch:\\
\begin{minipage}[t]{.45\linewidth}
{\bf Kinetische Limitierung}
\[ \frac{1}{k_s} \gg \frac{\delta}{D} \]
Schnelle Diffusion
\bi
\item Hohe Gasgeschwindigkeit
\item Niedrige Temperatur
\item Geringer Druck
\ei
\end{minipage}
\hfill
\begin{minipage}[t]{.45\linewidth}
{\bf Massentransportlimitierung }
\[ \frac{1}{k_s} \ll \frac{\delta}{D} \]
Langsame Diffusion
\bi
\item Geringe Gasgeschwindigkeit
\item Hohe Temperatur
\item Hoher Druck
\ei
\end{minipage}

\subsection{Zusammenfassung}
Der CVD/CVI-Prozess muss durch die Reaktion gesteuert werden, um Kontrolle �ber die Vollst�ndigkeit der Reaktion zu bekommen. Hier kommt es auf das Verh�ltnis von Depositions- und Diffusionsrate an.


\section{Ideale Reaktoren}
Allgemeine Massenbilanz:
\[ \fbx{ \underbrace{ \frac{\partial c_i}{\partial t} }_{ \mbox{Akk.} } = \underbrace{-\div \left( c_i \bar u \right)}_{ \mbox{Konvektion} }
+ \underbrace{\div \left( D_{e,i} \grad c_i \right)}_{ \mbox{Dispersion} } + \underbrace{\sum_j \nu_{ij} r_j}_{ \mbox{Reaktion} } } \qquad \left[ \frac{kmol}{m^3s} \right] \]

\subsection{Ideales Str�mungsrohr}
\[ \fbx{\frac{\partial c_i}{\partial t}=-u \frac{\partial c_i}{\partial z} + \Gamma_i} \qquad DaI=\frac{\tau r_0}{c_{10}}=\int^{X_i}_{0} \frac{r_0}{r} dX_i \]
mit $r=k c^n$ und volumenkonstanter Reaktion:
\[ \frac{r_0}{r}= \frac{k c^n_{10}}{c^n_1} = \frac{1}{\left(1 - X \right)^n} \]
f�r $n=1,2,3...\ \Rightarrow$ hyperbolische Funktion; f�r $X \rightarrow 1 \ \Rightarrow \frac{r_0}{r} \rightarrow \infty$ \\
Fl�che unterhalb Hyperbel entspricht $DaI$.
\bi
\item[$\Rightarrow$] Reaktionen h�herer Ordnung ben�tigen gr��ere $DaI$-Zahlen (l�ngere Verweilzeit oder gr��eres Reaktorvolumen, um den selben Umsatzgrad zu erreichen.
\ei
\paragraph{Spezialf�lle}
\bi
\item {\bf reversible Reaktion} Asymptote sinkt von $X=1$ auf $X=X_{eq}$
\item {\bf autokatalytische Reaktion} $r/r_0$-Kurve weist Minimum ($r=r_{max}$) auf
\ei

\subsection{Idealer kontinuierlicher R�hrkessel}
\[ \fbx{V \cdot \frac{dc_i}{dt}=\dot V \left( c_{i0} - c_i \right) + V \Gamma_i} \qquad  X_i = \frac{V r}{\dot V c_{i0}}\frac{r_0}{r_0} = DaI \frac{r}{r_0} \]
Rechteck zwischen $X$-$r/r_0$-Schnittpunkt und Achsen entspricht $DaI$.\\
$\Rightarrow$ Gew�hnlich ist $DaI_{CSTR}>DaI_{PFTR}$.\\
Im Falle autokatalytischer Reaktionen erreicht man den h�chsten Umsatz durch Reihenschaltung von CSTR und PFTR.

\paragraph{Sonderfall: enzymkatalisierte Reaktion}
\bi
\item hyperbolischer Geschwindigkeitsansatz
\item Massenstrom linear mit $X$, Reaktionsgeschwindigkeit nicht linear mit $X$
\item[$\Rightarrow$] mehrere stabile Betriebspunkte im $r$-$X$-Diagramm
\ei

\subsubsection{Enthalpiebilanz f�r CSTR}
\[ \fbx{ \underbrace{ \frac{\partial \left( \rho c_pT \right)}{\partial t} }_{ \mbox{Akk.} } =
\underbrace{-\div \left( \rho c_p T \bar u \right)}_{ \mbox{Konvektion} } + \underbrace{\div \left( \lambda_e \grad T \right)}_{ \mbox{eff. W�rmeltg.} } +
\underbrace{\sum_j \left( -\Delta H_R \right)_j r_j}_{ \mbox{Reaktion} } } \qquad\left[\frac{kJ}{m^3s}\right]  \]
Dimensionslose W�rmebilanz:
\[ \fbx{\underbrace{\left( 1 + H_W \right)}_{\mbox{W�rmetransfer}} \underbrace{ \left( \Theta - \Theta_A \right)}_{\mbox{K�hldifferenz}} = \underbrace{\frac{\tau q_R}{\rho c_p T_0}}_{\mbox{W�rmequelle}}} \]
mit dimensionslosen Kennzahlen:
\begin{eqnarray}
\mbox{W�rmetransferzahl:} & & \quad H_W = \frac{k_w F_k}{\dot V \rho c_p} \nonumber \\
\mbox{(K�hl-) Temperaturerh�hung:} & & \quad \Theta = \frac{T}{T_0}; \quad \left( \Theta_K = \frac{\bar T_K}{T_0} \approx 1,0 \right) \nonumber \\
\mbox{Eingangs- zu K�hltemperatur:} & & \quad \Theta_A = \frac{T_0 + H_W \bar T_K}{\left( 1 + H_W \right) T_0} \nonumber
\end{eqnarray}
Annahme $n=1$ und ersetzen von $X$ durch 
\[ X = \frac{DaI}{1 + DaI}=\frac{\tau k_0 \exp \left( - \gamma / \Theta \right)}{1 + \tau k_0 \exp \left( - \gamma / \Theta \right)} \]
f�hren zu:
\[ \fbx{\underbrace{\left( 1 + H_W \right) \left( \Theta - \Theta_A \right)}_{\mbox{W�rmeabfuhrgerade}} = \underbrace{\frac{\Delta T_{ad}}{T_0} \frac{\tau k_0 \exp \left( - \gamma / \Theta \right)}{1 + \tau k_0 \exp \left( - \gamma / \Theta \right)}}_{\mbox{W�rmeerzeugungskurve}}} \]

\paragraph{2 F�lle}
\be
\item $H_W$ variiert $\Rightarrow$ Steigung der WAG variiert ($\Theta_K = const.$).
\item $\Theta_K$ und somit $\Theta_A$ variiert $\Rightarrow$ WAG wird parallelverschoben ($H_W = const.$).
\item[$\Rightarrow$] Somit ist es m�glich den gew�nschten Betriebspunkt einzustellen.
\ee
\bi
\item {\bf isothermer Fall} $H_W \rightarrow \infty $: Steigung: $\infty$; WAG$=0$: $\Theta = \Theta_K$ 
\item {\bf allgemeiner Fall} Steigung: $1 + H_W$; WAG$=0$: $\Theta = \Theta_A$ 
\item {\bf adiabater Fall} $H_W = 0$: Steigung: $1$; WAG$=0$: $\Theta = \Theta_A = 1$ 
\ei
Polpunkt liegt in ($\Theta_K; \ \Theta_K - 1$); Grenzfall $T_0=\bar T_K \ \Rightarrow \ \Theta_K=1$: Pol($1; \ 0$)\\
\bi
\item[$\Rightarrow$] Vergleiche Z�nd-/L�schverhalten (Kernfachvorlesung)
\item Kriterium f�r stabilen Betriebspunkt:
\ei
\[ \mbox{Steigung}_{WAG} \stackrel{!}{>} \mbox{Steigung}_{WEK} \]
\bi
\item[PFTR:] nur ein Abschnitt befindet sich im kritischen Fall im z�ndwilligen Bereich von $X$.
\item[CSTR:] wenn $X$ in z�ndwilligen Bereich kommt, dann geht der ganze Reaktor durch!
\ei
\subsubsection{Instation�re Betriebsweise}
\bi
\item beim Hochfahren des Reaktors bis zum gew�nschten Betriebspunkt
\item beim �ndern des Betriebszustandes
\item[$\Rightarrow$] Wie lange dauert die instation�re Periode?
\ei
Aus instation�rer und station�rer Massenbilanz folgt f�r die Einstellung von $c_{A,s}$:
\[ t = -\frac{\tau}{ 1 + k \tau} \ln \left( \frac{c_{A,s} - c_A}{c_{A,s}} \right) \]
$\Rightarrow$ wichtiges Kriterium ist hier die Verweilzeit!



\section{Reale Reaktoren}
Reale Reaktoren unterscheiden sich von idealen Reaktoren in vielen Details:
\bi
\item Radiale Profile von Konzentration und Temperatur
\item Axiale Dispersion
\item Kanalbildung
\item Totvolumina
\item Kurzschlussstr�mungen
\ei
Eine hohe Aussagekraft bei der Modellierung realer Reaktoren bieten Verweilzeitverteilungskurven.\\
\bild[width=70mm]{images/vwz-summe.eps}

Fortgeschrittene Modelle ber�cksichtigen auch weitere Parameter.

\subsection{Segregation}
Die Segegration beschreibt das sogenannte Mikrovermischungsverhalten von Reaktoren. Man unterscheidet zwischen:
\bi
\item {\bf Mikrofluid} ($J = 0$) Vermischung auf mikroskopischer Ebene: Einzelne Molek�le sind beteiligt.
\item {\bf Makrofluid} ($J = 1$) Vermischung auf makroskopischer Ebene: Molek�lverb�nde bzw. Volumenelemente sind beteiligt.
\ei
Systeme, welche weder Makro- noch Mikrofluid sind, nennt teilsegregierte Fluide.
Segegration ist eine charakteristische Funktion von Fluid- und Systemeigenschaften.\\
Vollst�ndige Segregation bedeutet:
\bi
\item Molek�le k�nnen sich lediglich mit anderen Molek�len im eigenen Volumenelement vermischen. Eine Vermischung mit Molek�len eines anderen Volumenelements ist nicht m�glich.
\item Einzelne Einheiten vermischen sich mit anderen. Dies bestimmt die Mischeigenschaften eines Systems.
\ei
Experimentell kann der Grad der Segregation nur schwer bestimmt werden. Allerdings interessiert dies auch in der Praxis nicht.\\
Hier ist lediglich relevant:
\bi
\item Wie wird der Umsatz beeinflusst?
\item Welche Art der Vermischung f�rdert welche Reaktion?
\item Wie gro� sind die Unterschiede?
\ei

\subsubsection{Fallunterscheidung}
\be
\item {\bf BSTR} Kein Einflu� von Segregation, da alle Elemente zu gleicher Zeit Reaktion beginnen und gleich reagieren. Somit �berall gleiches $c$ bzw. $X$.
\item {\bf PFTR} Ebenfalls kein Einflu� der Segregation.
\item {\bf Alle anderen} Reaktortypen, in welchen nicht alle Volumenelemente die gleiche Verweilzeit besitzen, beeinflusst die Segregation den Umsatzgrad!
\item Bei {\bf Reaktionen 1. Ordnung} hat die Segregation wiederum keinen Einfluss!
\ee

\subsubsection{Berechnungen}
Im Fall einer Reaktion von A zu Produkten ergibt sich aus dem Potenzansatz bei Betrachtung zweier Volumenelemente ' und " f�r
\be
\item Micromixing
\[ r_M = k \left( \bar{c}_A \right)_M^n = k \frac{( c'_A + c"_A )^n}{2^n} \]
Hierbei ist $\bar{c}_A = ( c'_A + c''_A ) / 2$, also der Mittelwert der Konzentrationen beider Elemente.
\item Makromixing 
\[ r_S = k \frac{{c'}_A^n + {c"}_A^n}{2} \]
Da die Volumenelement getrennt sind, ergibt sich die Reaktionsgeschwindigkeit aus den Mittelwerten der Einzelnen.
\ee
Zum Vergleich bildet man das Verh�ltnis:
\[ \fbx{\frac{r_S}{r_M} = \frac{\frac{{c'}_A^n + {c"}_A^n}{2}}{\frac{( c'_A + c"_A )^n}{2^n}}} \]
Hier kann man 3 F�lle unterscheiden:
\be
\item $n = 1 \leadsto r_S/r_M = 1 \rightarrow$ kein Einfluss der Segregation auf den Umsatz.
\item $n > 1 \leadsto r_S/r_M > 1 \rightarrow$ h�herer Umsatz bei vollst�ndiger Segregation.
\item $n < 1 \leadsto r_S/r_M < 1 \rightarrow$ h�herer Umsatz bei Micromixing.
\ee
Daraus folgt direkt eine bestimmte Wahl der Betriebsweise!

\subsubsection{Modell idealer paralleler Str�mungsrohre}
Ein Reaktor mit bekannter Verweilzeitverteilung $E(t)$ kann mit dem Modell idealer paralleler Str�mungsrohre unterschiedlicher L�nge (Verweilzeit) angen�hert werden. Dabei geht man von kompletter Segregation der einzelnen Volumenelemente aus. 
\[ \left[ \bar X \right] = \sum \left[ \mbox{X f�r Element mit VWZ $t$} \right] \cdot \left[ \mbox{Anteil Elemente mit VWZ $t$ bis $t+dt$} \right] \]
\[ \fbx{ \bar{X} = \int_0^{\infty} X(t) \, E(t) \, dt = \int_0^1 X(t) \, dF(t)} \]
Das ben�tigte $X(t)$ stammt entweder aus einem kinetischen Ansatz oder wird im Batch-Reaktor experimentell bestimmt.

\subsubsection{Abh�ngigkeit der Selektivit�t einer Parallelreaktion von $J$}
F�r eine Parallelreaktion von A$_1$ zu A$_2$ mit $k_1c_1$ und zu A$_3$ mit $k_2c^2_1$ gilt:
\bi
\item In einem CSTR gleicher VWZ-Verteilung wird die Reaktion zweiter Ordnung durch vollst�ndige Segregation favorisiert.
\item Die Selektivit�t f�r A$_2$ wird folglich durch Micromixing erh�ht. 
\ei

\subsubsection{Graphische Methode zur Bestimmung des Umsatzgrades}
Die Zuordnungsmethode nach {\sc Sch�nemann} und {\sc Hofmann} bietet die M�glichkeit, bei bekanntem Verweilzeitverhalten und Kinetik eine Aussage zu dem durchschnittlichen Umsatzgrad zu machen.
\begin{minipage}[t]{.55\linewidth}
Das Prinzip ist:
\be
\item Aufzeichnen der experimentell bestimmten $F(t)$-Kurve des Reaktors
\item Aufzeichnen der experimentell bestimmten $X(t)$-Kurve der Reaktion aus Batch-Reaktor
\item $\displaystyle \bar{X} = \int_0^1 X(t) \, dF(t)$
\ee
\end{minipage}
\begin{minipage}[t]{.45\linewidth}
\bild[width=38mm]{images/xft.eps}
\end{minipage}
Dies bietet eine exakte L�sung f�r vollst�ndige Segregation und Reaktionen 1. Ordnung, sowie eine gute N�herung bei kleinen Umsatzgraden.
Die Grundlage liefert das Modell idealer paralleler Str�mungsrohre.

\subsection{Eindimensionales Dispersionsmodell}
Das Dispersionsmodell ist von hoher praktischer Relevanz zur Beschreibung realer Rohrreaktoren. Ausgehend von der Massenbilanz f�r einen station�ren, isothermen Reaktor mit volumenkonstanter, simpler Reaktion und lokal konstanter Dispersion folgt:
\[ \fbx{ u \frac{\partial c}{\partial z} = - D_z\frac{\partial^2c}{\partial z^2} + \sum \nu_{ij} r_j } \]
Durch Einsetzen der Randbedingungen nach {\sc Danckwert} ergibt sich:
\begin{eqnarray}
z = 0 &\quad& u c_i - D_z \frac{dc_i}{dz} = u c_{i0} \nonumber \\
z = L &\quad& \frac{dc_i}{dz} = 0 \nonumber 
\end{eqnarray}
Im Falle der analytischen L�sung erh�lt man 2 weitere Kennzahlen:
\begin{eqnarray}
\mbox{Axiale {\sc Peclet}-Zahl:} &\quad& Pe_{ax}=\frac{L \bar u}{D_{ax}} \nonumber\\
\mbox{Reaktionszahl $a$:} & \quad& a = \sqrt{1 + k \tau Pe_{ax}} \nonumber 
\end{eqnarray}
Es k�nnen folgende Aussagen gemacht werden:
\bi
\item Gro�e Unterschiede zwischen verschiedenen Reaktoren gibt es  bei hohen Ums�tzen (z.B. $X > 0.9$)
\item F�r Ums�tze $X \ge 0.99$ braucht man sehr hohe $DaI$
\item Bei gleicher $DaI$ erzielt ein PFTR stets den h�chsten Umsatzgrad
\ei

\subsubsection{Enthalpiebilanz}
In strikter Analogie zur Massenbilanz:
\[ \fbx{ u \frac{\partial T}{\partial z} = - \lambda_{ez}\frac{\partial^2T}{\partial z^2} + q_R - q_W} \]
\[ \mbox{mit} \quad q_R = \sum^M_{j=1} \left(- \Delta H_R \right)_j r_j \quad \mbox{und} \quad q_W = \frac{1}{d_t} k_W \left( T-T_K \right) \] 
Randbedingungen analog Massenbilanz.
Die effektive W�rmeleitf�higkeit $\lambda_{ez}$ kann �ber Korrelationen bestimmt werden.

\subsection{Mehrdimensionales Dispersionsmodell}
In der Praxis werden gew�hnlich 2-dimensionale Dispersionsmodelle verwendet bei
\bi
\item gro�em Reaktorradius $R$
\item schnellen Reaktionen mit hoher W�rmeausbeute
\item schlechter W�rmeleitf�higkeit in Festbetten
\ei
Den Bilanzraum stellt hierbei ein Kreisring-Volumenelement der L�nge $\Delta z$ und Dicke $\Delta r$ dar.
Bedingungen:
\bi
\item mehrere Reaktionen $j=1,\ldots,M$
\item pseudo-homogenes System von Fluid und Katalysator
\item Nicht konvektive Transportprozesse weren durch nur eine Richtung beschrieben: $D_z; \ D_r; \ \lambda_{ez}; \ \lambda_{er}$
\item Durchfluss ist �ber gesamten Reaktorquerschnitt konstant; $u_{Packung}=u_{Leerrohr}/\varepsilon$
\item W�rmetransport in radialer Richtung setzt sich aus $\lambda_{er}$ und $\alpha_{Wand}$ zusammen
\item radiale Symmetrie
\ei
Die Bilanz setzt sich wie folgt zusammen:
\[ \frac {\partial c_i}{\partial t}\underbrace{2 \pi r \Delta r \Delta z}_{\Delta V} = \mbox{ (1) - (2) + (3) - (4) + (5)} \quad \left[ \frac{mol}{s} \right] \]
\be
\item axiale Konvektion und Dispersion {\bf "Inflow"} an der Stelle $z$
\item axiale Konvektion und Dispersion {\bf "Outflow"} an der Stelle $z + \Delta z$
\item radiale Dispersion {\bf "Inflow"} an der Stelle $r$
\item radiale Dispersion {\bf "Outflow"} an der Stelle $r + \Delta r$
\item Reaktion im Volumenelement
\ee
Die W�rmebilanz kann wiederum in strikter Analogie behandelt werden.\\
Randbedingungen:
\be
\item $r=0$: kein radialer $c$/$T$-Gradient aufgrund Symmetrie
\item $r=R$: kein radialer $c$-Gradient, jedoch W�rme�bergang 
\item $z=0$: keine Reaktion, nur Dispersion und Konvektion
\item $z=L$: kein axialer $c$/$T$-Gradient aufgrund Quasistationarit�t 
\item[$\Rightarrow$] wird die Dispersion gegen�ber Konvektion vernachl�ssigt vereinfacht sich (3.) entsprechend  
\ee

\subsection{Kaskadenmodell}
Oft ist es schwierig ein {\it Randwertproblem}, wie das Dispersionsmodell, mathematisch zu l�sen.
Im Gegensatz dazu stellt das Kaskadenmodell gew�hnlich ein einfaches {\it Anfangswertproblem} dar.
Das Kaskadenmodell basiert auf einer R�hrkesselkaskade mit einem Kessel f�r den CSTR und unendlich vielen Kesseln f�r den PFTR.

\subsubsection{Eindimensionales Kaskadenmodell mit Recycle}
Das lineare Kaskadenmodell wird um einen Recycling-Strom $f$, der die axiale R�ckvermischung modelliert, erweitert.\\
Die Massenbilanz f�r den $k$-ten Kessel lautet:
\[ V_k \frac{d c_k}{d t} = \underbrace{ \dot V \left( 1 + f \right) \left( c_{k-1} - c_k \right)}_{ \mbox{Konvektion} } + \underbrace{\dot V f \left( c_{k+1} - c_k  \right)}_{ \mbox{Dispersion} } + \underbrace{V_k R_k}_{ \mbox{Reaktion} } \]

\subsubsection{Zweidimensionales Kaskadenmodell ohne Recycle}
Das zweidimensionale Kaskadenmodell modelliert einen Reaktor wie folgt:
\bi
\item $K$ Ebenen von jeweils $W/2$ {\it radial versetzt} �bereinander angeordenten Kreisringscheiben
\item Bef�llung jeweils durch zwei vorgeschaltetete Volumenelemente
\item Volumenelement ideal durchmischt
\item Entleerung jeweils in die zwei nachgeschalteten Volumenelemente
\ei
Im station�ren Fall werden nun $K \cdot W/2$, im allgemeinen nichtlineare, Gleichungen erhalten. Die gleichzeitige L�sung liefert die axiale und radiale Konzentrationsverteilung.

\section{Wirbelschicht-Reaktoren}
Bei Wirbelschichten unterscheidet man zwischen 6 Stadien (vgl. {\sc MVT}):
\be
\item $u < u_{mf}$: Festbett
\item $u \approx u_{mf}$: Lockerungszustand
\item $u > u_{mf}$: Blasenbildende Wirbelschicht
\item $u > u_{mf}$: Sto�ende Wirbelschicht
\item $u > u_{mf}$: Kanalbildende Wirbelschicht
\item $u \gg u_{mf}$: Feststoffaustrag - Zirkulierende Wirbelschicht
\ee
Der charakteristische Verlauf des Druckverlusts �ber der Str�mungsgeschwindigkeit erm�glicht die Bestimmung von $u_{mf}$.
�ber die H�he des Bettes am Lockerungspunkt kann $\varepsilon_{mf}$ bestimmt werden: 
\[ \varepsilon_{mf} = \frac{V_{s}}{V_{ges}} = A \frac{H_{\mbox{Festbett}}}{H_{\mbox{Lockerungspunkt}}} \]
Zur Beschreibung einer Wirbelschicht bedient man sich der Kr�ftebilanz:
\[ \fbx{ \Delta p = (1 - \varepsilon_{mf}) H g (\rho_p - \rho_f) } \]
Technisch unterscheidet man zwischen:
\bi
\item {\bf Homogene} Wirbelschicht: gleichm��ig expandiert mit $u < 3 u_{mf}$; Typ 2 und 3 (s.o.); Bsp. katalytische Gas-Phasen-Reaktion
\item {\bf Heterogene} Wirbelschicht: hoch expandiert mit $u > 3 u_{mf}$; Typ 4, 5 und 6; Bsp. Kohleverbrennung oder Rauchgasreinigung.
\ei

\subsection{Homogene Wirbelschichten}
Die wichtigste Eigenschaft der homogenen Wirbelschicht ist die Isothermie. Deswegen kann auf eine Enthalpiebilanz komplett verzichtet werden.
Ausgehend vom eindimensionalen Dispersionsmodell kann nun die Massenbilanz aufgstellt werden:
\[ \fbx{0 = - u \frac{dc_1}{dh} + D_a \frac{d^2c_1}{dh^2} + k c_1 } \]
Hierbei ist zu bemerken, dass $D_a = f(d_p;\Psi)$, also von Partikelgr��e und Beschaffenheit (Spherizit�t) abh�ngig ist.

\subsection{Heterogene Wirbelschichten}
F�r eine heterogen fluidiesierte Wirbelschicht ist es schwer, eine angemessene Beschreibung f�r den Massentransport zwischen Gasblasen und Katalysator-{\emph Suspension} zu finden. Hier muss auf Modelle zur�ckgegriffen werden:

\subsubsection{Blasenmodell}
Dieses Modell wurde von {\sc Kunii} und {\sc Levenspiel} entwickelt und beschreibt den Massentransport zwischen Blase und Suspension in zwei Schritten:
\be
\item Massentransport Blase$\rightarrow$Wolke $K_{bc}$
\item Massentransport Wolke$\rightarrow$Suspension $K_{cs}$
\ee
Man beschreibt eine aufsteigende Blase hierbei als einen Batch-Reaktor mit Reaktion 1. Ordnung:
\[ 0 = - u_b \frac{dc_b}{dh} - K_{eff} c_b \]
Dabei vereinigt $K_{eff}$ alle kinetischen und massentransportbehafteten Effekte.
Zur Bestimmung von $K_{eff}$ trifft man folgende Annahmen:
\bi
\item Jegliches Gas steigt in Form von Blasen durch die (sozusagen autark) fluidisierte Wirbelschicht
\item Alle Gasblasen verhalten sich gleichartig in Bezug auf Verweilzeit, Gr��e und Form
\item Der Gasaustausch erfolgt �ber die Wolke/Schleppe der Blase
\item Stationarit�t
\ei
Hieraus ergibt sich folgendes L�sungskonzept:
\be
\item Aufstellen von Bilanzgleichungen f�r Blase B, Wolke W und Suspension S:
\ee
\begin{eqnarray}
\mbox{Gesamtumsatz} &=& \mbox{Reaktion in B} + \mbox{Transport B-W} \\
\mbox{Transport B-W} &=& \mbox{Reaktion in W} + \mbox{Transport W-S} \\
\mbox{Transport W-S}&=& \mbox{Reaktion in W}
\end{eqnarray}
\bi
%Reaktion in Blase = Transport von Wolke\\
%Transfer von Wolke = Reaktion in Wolke + Transport von Suspension\\
%Transport von Suspension = Reaktion in Suspension
\item[2.] L�sen dieser durch Einsetzen von (3) in (2) und (2) in (1).
\item[3.] Eliminieren der unbekannten Konzentrationen $c_s$ und $c_c$.
\item[4.] Aufl�sen nach $K_{eff}$.
\ei
Nach L�sung der Massenbilanz
\[ (1-X) = \exp \left( - \frac{K_{eff} H}{u_b} \right) \]
kann $1-X$ in Abh�ngigkeit von $H$ grafisch aufgetragen werden.\\
$\ominus$ Der unsicherste Parameter ist hierbei $u_b$, welcher abh�ngig von der Blasengr��enverteilung aufgetragen wird.
\bi
\item experimentelle Ermittlung
\item (nachtr�gliche) Anpassung der Parameter
\item[$\Rightarrow$] kein A-Priori-Modell!
\ei

\subsubsection{Zweiphasenmodell}
In diesem Modell von {\sc Van Deemter} wird angenommen, dass sich das einstr�mende Gas in zwei Teile aufspaltet:
\be
\item Teil zum Fluidisieren des Betts. Str�mt durch Suspensionsphase mit $u_{mf}$.
\item Teil formt Blasen, welche mit $u - u_{mf}$ durch den Reaktor wandern.
\ee
Zwischen Blase und Suspension findet ein Massentranfer $\alpha$ statt.
Eine Reaktion findet ausschlie�lich in der Suspension statt (hier ist der Unterschied zum Blasenmodell!).\\
Massenbilanz f�r Suspensionsphase:
\[ 0 = - u_{mf} \frac{dc_s}{dh} + D_a \varepsilon_{mf} \frac{d^2c_s}{dh^2} + \alpha (c_b -c_s) + (1-\varepsilon_b)(1-\varepsilon_{mf}) \rho_s k' c_s \]
Massenbilanz f�r die Blase-Phase:
\[ 0 = - (u - u_{mf}) \frac{dc_b}{dh} - \alpha (c_b - c_s) \]

\subsubsection{Filmmodell}
{\sc Werther} entwickelte als Erweiterung des Zweiphasenmodells das Filmmodell, welches eine Behandlung analog der Fluid-Fluid-Reak\-tionen vorschl�gt.
\bi
\item Bilanzierung um Blase, Film und Suspension
\item Massentransport durch die Phasengrenze wird einer Adsorption gleichgesetzt
\item pseudo-homogene Reaktion in der Suspensionsphase wird angenommen
\ei
Durch die �blichen Randbedingungen erfolgt eine analytischen L�sung, welche, wie bereits aus der Fluid-Fluid-Beschreibung bekannt, die $Ha$-Zahl und $B$ enth�lt.
\bi
\item[$\oplus$] Zug�nglichkeit von $k_g$, $a$ (spez. Oberfl�che in $B$) und $\varepsilon_{mf}$ �ber Korrelationen
\item[$\Rightarrow$] {\bf A-Priori-Design} von Wirbelschichten ohne Experimente ist m�glich!
\ei

\section{Reaktionsf�hrung}
\subsection{Maximale Produktivit�t in einem Batch-Reaktor}
In einem Batch-Reaktor ($n > 0$) nimmt die Reaktionsgeschwindigkeit mit zunehmer N�he zum Gleichgewichtsumsatz ab.\\
Deswegen stellt sich die Frage nach dem optimalen Umsatz $X_{opt}$ und der optimalen Reaktionszeit $t_{opt}$.
Produktivit�t ergibt sich zu:\\
\begin{minipage}[t]{.4\linewidth}
\[ \fbx{L_p = \frac{c_{10} V X(t)}{t+t_{opt}} } \]
Somit ergibt sich:
\[ \left( \frac{dX}{dt} \right)_{opt} = \frac{X_{opt}}{t+t_{opt}}\]
\end{minipage}
\begin{minipage}[t]{.6\linewidth}
\bild[width=50mm]{images/mincosts.eps}
\end{minipage}
Im Diagramm ist diese leicht grafisch bestimmbar:
\[ \left( \frac{dX}{dt} \right)_{cost} = \frac{X_{cost}}{t + \frac{K_3 + K_2 t_{tot}}{K_1}} \]

\subsection{Minimale Kosten in einem Batch-Reaktor}
Die Kosten f�r den Betrieb eines Batch-Reaktors setzen sich zusammen aus:
\bi
\item $K_1$ zeitabh�ngige Kosten w�hrend des Betriebs
\item $K_2$ zeitabh�ngige Kosten w�hrend der Reinigung
\item $K_3$ Feste Kosten pro Ansatz
\ei

\subsection{Optimale Temperaturf�hrung}
Die optimale Temperaturf�hrung h�ngt von der Art der Reaktion ab:
\bi
\item {\sc Irreversible} Reaktionen: Die h�chste Reaktionsgeschwindigkeit wird bei h�chster Temperatur erreicht. In so fern sollte die Reaktion bei h�chster wirtschaftlich sinnvoller Temperatur ablaufen.
\item {\sc Endotherme} Reaktionen: Durch hohe Temperatur wird die Reaktionsgeschwindigkeit und das Gleichgewicht g�nstig verschoben. Somit auch hier: hohe Temperatur.
\item {\sc Exotherme} Reaktion: Durch niedrige Temperatur wird das Gleichgewicht g�nstig verschoben, aber die Reaktionsgeschwindigkeit nimmt ab. Hier liegt ein Optimierungsproblem vor
\ei
Die optimale Temperatur liegt bei $\frac{\partial R_{Produkt}}{\partial T} = 0$.\\
Zur L�sung f�r ein konkretes Problem geht man von $R$ aus und l�st nach Einsetzen von {\sc Arrhenius} und Einf�hren des Umsatzgrades nach $T$ auf.

\subsection{Komplexe Reaktionen}
Hier bieten sich 3 M�glichkeiten an, um die Selektivit�t zu verbessern:
\be
\item Einsatz eines selektiven Katalysators bzw. L�semittels
\item Temperaturkontrolle (vgl. {\sc Wheeler})
\item Konzentrationskontrolle (vgl. folgende Kapitel)
\ee

\subsubsection{Parallelreaktionen}
Allgemein gilt:
\bi
\item PFTR f�rdert Reaktionen h�herer Ordnung
\item CSTR f�rdert Reaktionen niedrigerer Ordnung
\ei
Die Selektivit�t ergibt sich hier zu:
\[ S = \frac{r_1}{r_2} = \frac{k_1}{k_2} c_1^{n-m} \]
\bi
\item $n > m$ - Hier sollte $c_1$ so hoch wie m�glich sein.
\item $n < m$ - Hier sollte $c_1$ so niedrig wie m�glich sein.
\item $n = m$ - Hier hat die Konzentration keinen Einfluss.
\ei
Sind mehrere Edukte beteiligt wird die Betrachtung schwieriger:
\[ S = \frac{r_1}{r_2} = \frac{k_1}{k_2} c_1^{n_1-m_1} c_2^{n_2-m_2} \]
Hierbei ergibt sich eine Matrix. Kurz gesagt: h�here Ordnung braucht h�here Konzentration.

\subsubsection{Folgereaktion}
F�r eine Folgereaktion wird ebenfalls optimale Ausbeute erw�nscht. Im Vergleich zwischen PFTR und CSTR ergibt sich aus:
\[ X_{PFTR} = 1 - \exp (-DaI) \qquad X_{CSTR} = \frac{DaI}{1+DaI} \]
eingesetzt in $\frac{c_2}{c_{10}}$, dass hier, unabh�ngig von $\frac{k_1}{k_2}$ stets beim PFTR bei geringerer $DaI$ eine gr��ere Selektivit�t vorliegt.

\section{Wirtschaftliche Betrachtung von Prozessen}
\subsection{Produktionskosten}
\bi
\item Rohstoffe, Additive, Katalysatoren, etc.
\item Energie (Elektrizit�t, Dampf, K�hlmittel, etc.)
\item Personalkosten (L�hne, Geh�lter, Pr�mien, etc.)
\item Kapitalkosten (Zinsen und Abschreibungen)
\item Erhaltungskosten
\item Umweltschutz (z.B. Abwasserbehandlung)
\item Analyse und Qualit�tsmanagment
\item Versand und Logistik
\ei

\subsection{Andere Kosten}
\bi
\item Generalia: Beteiligung an Gesamtkosten des Unternehmens
\item Marketing: technisches Marketing, Publikationen, Werbung
\item Forschung und Entwicklung: Investition in die Zukunft!
\ei

\subsection{Kalkulation}
Preise f�r Hauptpositionen (Reaktor, W�rme�bertrager, etc.) sind normalerweise bekannt.
Sind die Kosten $P$ f�r gew�nschte Gr��e oder Kapazit�t $C$ nicht bekannt kann die Korrelation
\[ \frac{P_1}{P_2} = \left( \frac{C_1}{C_2} \right)^m \]
mit einem Degressionskoeffizienten von z.B. $m=0,66$ zur Hochrechnung herangezogen werden.\\
Direkte Nebenpositionen (Montage, Verrohrung, MSR-Technik) und indirekte Nebenpositionen (Planungskosten, unerwartete Ereignisse) werden mit Faktoren abgesch�tzt und betragen in etwa das $1,9$ bis $3,5$-fache der Hauptpositionen.

\subsection{Kostendiagramm}
Linearer oder nichtlinearer Auftrag von Erl�sen bzw. Kosten [$EUR/a$] gegen�ber Auslastungsgrad $X$.
\bi
\item fixe Kosten: Kapital, Personal...
\item variable Kosten: Rohstoffe, Energie...
\item Nutzenschwelle: Erl�se > fixe + variable Kosten
\item Stillegungspunkt: variable Kosten > Erl�se
\ei

\subsection{Kapitalrendite und Amortisationszeit}
Der Parameter {\it return on investment} $r$ (ROI) basiert auf der Annahme, dass Profit und Abschreibungen verwendet werden, um die Investitionskosten zur�ckzuzahlen.
\[ r = \frac{ \mbox{Profit}/a+ \mbox{Abschreibungen}/a }{ \mbox{investiertes Kapital} } \cdot 100 = \frac{1}{t_r} \quad \left[ \%/a \right] \]
Wird f�r die Finanzierung Eigenkapital anstelle von Fremdkapital herangezogen, werden im Z�hler noch die j�hrlichen Zinsen hinzugerechnet.
In der chemischen Industrie sind gew�hnlich Amortisationszeiten $t_r$ von weniger als vier Jahren gefordert, um die wirtschaftlichen und technischen Risiken zu rechtfertigen. Nur f�r gut etablierte Massenprodukte mit einem sicheren, wachsenden Markt sind l�ngere Amortisationszeiten m�glich.

\section{Prozessentwicklung}
Unterschiede in der Prozessentwicklung f�r Massen- und Feinchemikalien sind durchg�ngig.\\
Scale-Up als empirischer Prozess ist zeitraubend und sicherheitstechnisch riskant, was zu einer Bevorzugung voraussagender Modelle f�hrt.
\subsection{Hauptschritte in der Prozessentwicklung}
\be
\item {\bf Entdeckungsphase} eines neuen Produktes, Katalysators, Syntheseweg
\item {\bf Verfahrenskonzept:} Machbarkeit, Wirtschaftlichkeit, Reaktorkonzept, Trennverfahren
\item {\bf Miniplant/Pilotanlage:} Entwicklung des Marktes, Best�tigung der Machbarkeit, Erforschung von Langzeiteffekten
\item {\bf Kommerzielle Anlage:} Detailplanung und Montage 
\ee
Zwischen den Hauptschritten steht eine Bewertung, die zum Verwerfen der Entwicklung, zur Wiederholung eines Schrittes oder zur Fortsetzung der Kette f�hrt.\\
Moderne Prozessentwicklung geschieht aus Zeitgr�nden oftmals parallel ({\it simultaneous engineering}).
Dies erfordert eine intensive Zusammenarbeit zwischen unterschiedlichen Disziplinen (Chemiker, Reaktionstechniker, Trenntechniker, etc.).

\section{Sicherheitsaspekte}
\subsection{Unfallursachen}
\begin{minipage}[t]{.45\linewidth}
\bi
\item mechanisches Versagen
\item Bedienungsfehler
\item Ursache unbekannt
\item Prozess-St�rung
\ei
\end{minipage}
\hfill
\begin{minipage}[t]{.45\linewidth}
\bi
\item Naturgewalt
\item Auslegungsfehler
\item Brandstiftung, Sabotage
\ei
\end{minipage}
\paragraph{Beispiel:} stark exotherme Oxidation von o-Xylen im PFTR\\
Partialdruckerh�hung von o-Xylen f�hrt zu:
\be
\item unerw�nschtem Hot-Spot
\item "Durchgehen" des Reaktors
\item[$\lightning$] kleine Schwankungen von Konzentration und Temperatur liegen im Bereich von Dosierungenauigkeiten.
\ee
Kritische Betriebsbereiche vermeiden!\\
$\Rightarrow$ Empfindlichkeitsdiagramm von {\sc Berkelew}:
\bi
\item dimensionsloser adiabater Temperaturanstieg $S$ ($\Delta H_R; \ E_A; \ T_0;...$)
\item dimensionslose W�rmeabfuhrzahl $N$ (W�rmeleitung, Reaktorgeometrie, Reaktion)
\item[$\Rightarrow$] Auftragung von $N/S$ gegen $S$:
\item oberhalb der Kurve: $S$ klein, $N/S$ gro�: unkritischer Betrieb
\item unterhalb der Kurve: $S$ gro�, $N/S$ klein: kritischer Bereich
\item Erh�hung von $T_0$ f�hrt zur Ausdehnung des kritischen Bereiches
\item Erh�hung der Reaktionsordnung f�hrt zur Ausdehnung des sicheren Bereiches
\ei
\subsection{Unfallfolgen}
%\begin{minipage}[t]{.27\linewidth}
%\be
%\item "Durchgehen" von Reaktionen
%\item Brand oder Explosion
%\\
%\item Freiwerden gef�hrlicher Stoffe
%\ee
%\end{minipage}
%\hfill
%\begin{minipage}[t]{.63\linewidth}
%\bi
%\item H�chst, 1993, Nitrochlorbenzol
%\item Flixborough, GB, 1974, Cyclohexan
%\item Oppau, BASF, 1921, Ammoniumnitrat
%\item Seveso, 1976, "Sevesodioxin" TCDD
%\item Bophal, Indien, 1984, Methylisocyanat
%\item Basel, Sandoz, 1986, kontaminiertes L�schwasser im Rhein
%\ei
%\end{minipage}
\bi
\item[1.] {\bf "Durchgehen" von Reaktionen:}
\item H�chst, 1993, Nitrochlorbenzol
\item[2.] {\bf Brand oder Explosion:}
\item Flixborough, GB, 1974, Cyclohexan
\item Oppau, BASF, 1921, Ammoniumnitrat
\item[3.] {\bf Freiwerden gef�hrlicher Stoffe:}
\item Seveso, 1976, "Sevesodioxin" TCDD
\item Bophal, Indien, 1984, Methylisocyanat
\item Basel, Sandoz, 1986, kontaminiertes L�schwasser im Rhein
\ei
\subsection{M�glichkeiten des Gefahrenschutzes}
\be
\item Umhausung, Abschirmung
\item MSR-Technik, Schnellschlussventile
\item eigensichere Prozesse
\ee
Ein Prozess ist eigensicher, wenn keine St�rung zu einem Unfall f�hren kann:
\bi
\item Identifizierung aller m�glichen Reaktionen
\item Minimierung der Menge gef�hrlicher Stoffe
\item Vermeidung aller externer Ausl�ser von "Runaways"
\item Betriebsbedingungen fern vom kritischen Bereich
\ei
\section{Nachhaltige Chemieproduktion}
Nachhaltige Entwicklung erf�llt die Bed�rfnisse unserer Generation auf eine Art und Weise, die auch zuk�nftigen Generationen erm�glicht, ihre Bed�rfnisse zu erf�llen.
\[ \mbox{Nachhaltigkeit = �kologie + �konomie + soziale Aspekte} \]
\subsection{"Gr�ne" chemische Prozesse}
\bi
\item Energie- und Atom-Effizienz
\item Katalysatoren
\item umweltfreundliche L�semittel
\item Prozess�berwachung und Sicherheit
\item alternative Rohstoffe
\item neue Produkte
\ei




\end{document}


